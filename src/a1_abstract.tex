% !TeX root = ../main.tex
\begin{abstract}

开源软件 (Open source software, OSS) 漏洞管理已然成为一个热点问题。开源漏洞数据库为解决漏洞问题提供十分有价值的数据信息,因此,漏洞数据库的数据质量也受到越来越多的关注和研究。具体的问题为:现有漏洞数据库中补丁的质量尚未研究清楚,此外,现有的补丁信息多由人工或基于启发式的识别方法进行收集。这种方法人工成本过高,且过于定制化无法应用于全部的OSS漏洞。

\congyingEdit{empirical study该如何翻译比价好呢?实证研究?经验性研究?}
为了解决这些问题,首先,我们进行了实证研究,以了解当前商业旗舰漏洞数据库中开源软件漏洞补丁的质量和特征。我们的研究涵盖五个方面,包括:补丁的覆盖度、一致性、类型、\congyingEdit{基数}和准确性。然后,基于研究的发现,我们提出了第一种名为 \tool 的自动化方法,用于从多个来源查找开源漏洞的补丁。%我们的主要思想是在报告、讨论和解决 OSS 漏洞期间会经常引用补丁提交。

实验评估表明:i) 与现有的基于启发式的方法相比,\tool 能够为多达 \tocheck{273.8\%} 的 CVE找到补丁;同时,准确性方面,将F1数值提高达 \tocheck{116.8\%};ii) 与现有的漏洞数据库相比,\tool 将召回率(recall)提高达 \tocheck{18.4\%} ;然而,\tocheck{12.0\%} 的 CVE补丁未找到,精度(precision)下降约\tocheck{6.4\%}。%我们的评估也证明了\tool 的通用性和实用性。

\end{abstract}
