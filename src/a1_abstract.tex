% !TeX root = ../main.tex
\begin{abstract}

开源软件(Open Source Software, OSS)漏洞管理已成为一个备受关注的研究课题。其中,开源软件漏洞数据库作为基础设施,

为解决相关的研究问题提供了非常有价值的漏洞信息


开源软件漏洞数据库对于开源软件漏洞管理来说十分重要,因为它为解决相关的研究问题提供了非常有价值的漏洞信息。

因此,漏洞数据库中的数据质量也受到了越来越多研究人员的关注和研究。然而,现有的漏洞数据库中补丁的质量和特征尚未被系统地研究。此外,漏洞数据库中的补丁也多是由人工或基于启发式规则的方法半自动化收集。这些方法人工成本过高,且任务定制化特征明显,无法应用于大范围的开软软件漏洞。

为了解决上述问题,
本文在两个包含数据集上,开展了一个关于开源软件漏洞补丁的

8630 5858

首先进行了一项经验研究,以了解当前商业漏洞数据库中开源软件漏洞补丁的质量和特征。该经验研究涵盖了漏洞补丁的五个方面,包括覆盖度、一致性、类型、映射关系和准确性。【列举经验研究的一些主要分析结果】

基于经验研究的发现,本文提出了一个名为\tool 的基于多源知识的开源软件漏洞的补丁识别方法。该方法通过构建针对漏洞的多源引用信息网络,以识别补丁。【在多少的实验数据集上,做了xxx实验】实验结果表明:(1)与现有的基于启发式规则的方法相比,\tool 能够多找到273.8\%漏洞的补丁;同时,在补丁的准确性上,\tool 的F1数值(F1-Score)也比基于启发式规则的方法高116.8\%;(2)与商业漏洞数据库相比,\tool 的召回率(Recall)高18.4\%;但12.0\%的漏洞\tool 未能找到补丁,\tool 的精确率(Precision)也低6.4\%;(3)\tool 对于大范围开源软件漏洞具有较好的通用性,在实际使用中,也具有较好的实用性。

\end{abstract}
