% !TeX root = ../main.tex
\begin{abstract}

开源软件(Open Source Software, OSS)漏洞管理已成为一个备受关注的研究问题。开源软件漏洞数据库作为各项软件安全任务的基础设施,数据库中漏洞知识的质量受到了越来越多的关注和研究。然而,现有的漏洞数据库中补丁知识的质量和特征尚未被系统地研究。此外,漏洞数据库中的补丁也多是由人工或基于启发式规则的方法半自动化收集。这些方法人工成本高并且为任务定制化,无法应用于所有开源软件漏洞。

针对上述问题,本文先开展了一项针对开源软件漏洞补丁的经验研究,以了解当前商业漏洞数据库中开源软件漏洞补丁的质量和特征。该经验研究涵盖五个方面,包括补丁覆盖度分析、补丁一致性分析、补丁类型分析、补丁映射关系以及补丁准确性分析。研究发现:(1)商业漏洞数据库中开源软件漏洞补丁的质量并不理想。补丁缺失情况较为普遍,商业数据库中漏洞的补丁覆盖率仅为41.0\%左右。对于有多个补丁的漏洞,商业数据库经常会遗漏部分补丁。(2)开源软件漏洞补丁在类型、映射关系方面有一定的特殊性。93.7\%的补丁类型都是GitHub的代码提交,超过40\%的漏洞与其补丁是一对多的映射关系。

基于经验研究的发现,本文提出了一种名为\tool 的基于多源知识的开源软件漏洞的补丁识别方法。该方法用于识别代码提交类型的补丁,并构建一对多的漏洞补丁映射关系。该方法的核心思想是:漏洞补丁会在讨论和解决漏洞的、多种来源的漏洞公告、分析报告等参考链接中被频繁地提及和引用。因此,本文首先设计了一种基于多知识源的漏洞参考链接网络,再从该网络中选出具有最高置信度和连通度的补丁节点作为结果,并基于选定的补丁进行补丁扩增,从而构建一对多的漏洞补丁映射关系。

本文通过五个研究问题,从准确性、通用性、实用性等多个方面对\tool 进行了实验评估。实验结果表明:(1)在包含1,295个漏洞的实验数据集上,\tool 可以达到88.0\%的补丁覆盖率、0.864的补丁精确率和0.864的补丁召回率。(2)与现有的基于启发式规则的方法相比,\tool 将补丁覆盖率提高\tocheck{58.6\%}到\tocheck{273.8\%},将F1值提高\tocheck{116.8\%}。(3)与商业漏洞数据库相比,\tool 的召回率高出18.4\%。这表明,\tool 可用于补充现有漏洞数据库缺失的漏洞补丁数据。(4)在更大范围包含\tocheck{3,185}和\tocheck{5,468}个漏洞的数据集上,\tool 具有较好的通用性。即使在商业漏洞数据库都没有补丁的情况下,\tool 仍能达到\tocheck{67.7\%}和\tocheck{51.5\%}的补丁覆盖率。这表明,\tool 可以极大地补充或增强现有的商业漏洞数据库。在实际工作场景中,\tool 有助于用户更准确、更快速地识别到补丁。
\end{abstract}