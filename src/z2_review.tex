\chapter{初审意见修改说明}

衷心感谢老师们的耐心评审,以及提出的宝贵修改意见。修改说明如下:

\textbf{初审1}\\
格式问题:
\begin{enumerate}[]
%   \item \strong{方法叙述上,应该多加论述为什么,而不仅是罗列过程。}\\
%   在第三章描述各个方案前补充了这么做的原因。
\item \strong{扉页(提交学院和学校盲审版时去掉指导小组成员信息) ×(缺页);}\\
  已补充“指导小组成员”信息页,并在提交学院和学校盲审版时将其屏蔽。
\item \strong{摘要  中英文摘要用罗马数字编页码,从I开始;}\\
  已重新编排页码,中文摘要页现从I开始。
\item \strong{正文中各级标题 不同级标题之间不要直接相连,其间用一段总括的句子隔开;}\\
  文中,子标题“2.1 背景知识”、“2.2 相关工作”、“3.1 研究设计”之前及“第六章 总结与展望”之后,已添加一段总括的句子。
\item \strong{独创性声明和论文使用授权声明(提交学院和学校盲审版时去掉这部分) 缺少该页}\\
  在提交学院和学校盲审版时,独创性声明和论文使用授权声明已屏蔽,未在论文中显示该页。盲审结束后,会补充该页。
\end{enumerate}

正文内容: 正文内容完整详细,逻辑通顺,但有一些小的细节问题。
\begin{enumerate}[]
    \item \strong{第九页 语句不通顺,‘出现了许多第三方平台致力于收集并维护所有漏洞公告信息’—>‘出现了许多致力于收集并维护所有漏洞公告信息的第三方平台’。}\\
      文中已更改为“出现了许多致力于收集并维护所有漏洞公告信息的第三方平台”。
    \item \strong{第十八页: 3.3 RQ1:补丁覆盖率分析各种比率写的比较混乱,例如对于包含10070个开源软件漏洞的广度,仅有1417个漏洞含有补丁(不是2190+1417+995?),补丁覆盖率为32\%。}\\
      此处为笔误,文中已更改为“对于包含 10,070 个开源软件漏洞的广度数据集,仅有4,602个漏洞含有补 丁,补丁覆盖率为 45.7\%。”。
    \item \strong{第十九页: 表3-1 无漏洞信息、无补丁(有歧义,应注释在某一数据库无漏洞信息、无补丁)。}\\
      表3-1中,已将“无漏洞信息“、”无补丁”分别更改为“某一数据库中无漏洞”、“某一数据库中无补丁”。
    \item \strong{第二十页: 语言不准确 RQ2:补丁一致性分析 1793(39\%)的漏洞都存在于DBA和DBB中但在某一数据库中无补丁--> 1793(39\%)的漏洞都存在于DBA或DBB中但在另一数据库中无补丁。}\\
      文中已更改为“1793(39\%)的漏洞都存在于DBA或DBB中但在另一数据库中无补丁”。
\end{enumerate}

\textbf{初审2}\\
文章针对当前漏洞研究中漏洞补丁管理中的不足,设计了相关统计实验进行研究总结,并根据经验研究结果针对已有漏洞补丁数据库的痛点,设计了 TRACER 工具来自动识别相关漏洞的漏洞补丁进行漏洞补丁数据库的补充,通过充足的实验数据证明了 TRACER 能有效提升漏洞补丁的覆盖率和准确率。在实用性、通用性等都表现了较高的实用价值。可应用于安全社区、工业和学术界。

优点:文章研究内容具有一定的创新性、可用性,具有一定的学术价值。文章结构组织合理,逻辑清晰,行文流畅。实验充分。引用丰富。
待改进地方: 
\begin{enumerate}[]
    \item \strong{本文缺少TRACER 测试时的硬件资源情况。例如运行的时间成本、物理环境或实现TARACER代码量等。}\\
      文中已添加章节“5.2 实验环境准备”,介绍了TRACER在测试时的硬件资源情况,包括运行的时间成本、物理环境及实现TRACER的代码量。具体内容为:“本文作者使用Python 3.6搭建实验平台、开发工具\tool ,\tool 中共包括6117行代码。实验评估中,\tool 运行在含有16核的Intel(R) Xeon(R) CPU @ 2.10GHz处理器、 64GB内存、Red Hat 4.8.5-39系统的服务器上。基于从广度数据集中随机抽样的200个漏洞,运行结果表明,\tool 在该实验环境下完成一次漏洞补丁定位的平均用时约为22分钟。”。
    \item \strong{部分标点格式不统一。例如: p53 中 6.1节 (3) 中冒号。}\\
      已调整p53中6.1节(3)中冒号的格式。
\end{enumerate}



