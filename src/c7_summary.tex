\chapter{总结与展望}

% 本章节总结与展望。

\section{本文总结}
开源软件(Open Source Software,OSS)加快了开源及闭源应用程序的开发进程,但也将安全风险引入了应用程序中。随着开源软件中被披露的漏洞越来越多,开源软件漏洞管理也逐渐成为一个热点问题,大量软件供应链安全相关的工作都是基于漏洞补丁知识开展。然而,开源软件漏洞数据库中补丁的特征及质量尚未被系统地研究和评估,并且已有的漏洞补丁采集方法通用性较差且人工成本过高。

为解决上述研究问题,本文先开展了一项针对开源软件漏洞补丁的经验研究,以了解当前商业漏洞数据库中开源软件漏洞补丁的质量和特征。该经验研究涵盖五个方面,包括补丁覆盖度分析、补丁一致性分析、补丁类型分析、补丁映射关系以及补丁准确性分析。研究发现:
\begin{enumerate}
    \item [(1)]商业漏洞数据库中开源软件漏洞补丁的质量并不理想,体现在:(i)开源软件漏洞补丁缺失情况较为普遍,商业数据库$DB_A$和$DB_B$中开源软件漏洞的补丁覆盖率仅为41.8\%和41.2\%。(ii)商业漏洞数据库$DB_A$和$DB_B$具有较高的精确率,但经常会遗漏一些漏洞的补丁,尤其是对于具有多个补丁的漏洞。对于安全服务用户来说,这会给漏洞的及时检测和修复带来较大困难。这体现出当前开源软件漏洞数据库的不足,以及利用自动化补丁识别方法完善漏洞数据的需求。
    \item [(2)]开源软件漏洞补丁在类型、映射关系方面有一定的特殊性,设计自动化补丁识别方法时应充分考虑。体现在:(i)93.7\%的补丁都为GitHub代码提交的形式。(ii)开源软件漏洞与其补丁之间映射关系具有多样性,超过40\%的漏洞与其补丁具有一对多的映射关系。
\end{enumerate}

基于经验研究的发现,本文提出了一种名为\tool 的基于多源知识的开源软件漏洞的补丁识别方法。该方法用于识别代码提交类型的补丁,并构建一对多的漏洞补丁映射关系。该方法的核心思想是:漏洞的补丁会在与该漏洞相关的各种来源的漏洞公告、分析报告、讨论和解决的过程中被频繁地提及和引用。因此,本文首先设计了一种基于多知识源的漏洞参考链接网络,再从该网络中选出具有最高置信度和连通度的补丁节点作为结果,并基于选定的补丁进行补丁扩展,从而构建一对多的漏洞补丁映射关系。

本文还进行了大量实验,通过五个研究问题,从准确性、通用性、实用性等多个方面对\tool 进行了评估。研究发现:
\begin{enumerate}
    \item [(1)] \textbf{准确性:}\tool 可以达到88.0\%的补丁覆盖率、86.4\%的精确率和86.4\%的召回率。与现有的基于启发式规则的方法相比,\tool 能够显著的提高补丁覆盖率和F1值。与商业漏洞数据库$DB_A$和$DB_B$相比,\tool 以略低的补丁精确率和覆盖率为代价,拥有更为显着的召回率。这表明,\tool 可用于补充现有漏洞数据库缺失的漏洞补丁数据。
    \item [(2)] \textbf{削弱性:}\tool 中的知识源、网络构建、补丁选择和补丁扩增步骤,及其中环节的设计对最终结果都有一定的贡献度和必要性。
    \item [(3)] \textbf{敏感度:}\tool 的准确率对两个可配置参数的变化不是非常敏感,且\tool 参数采用默认设置时效果相对最优。
    \item [(4)] \textbf{通用性:}在更大范围的开源软件漏洞上,\tool 具有较好的通用性,覆盖率和准确率比较稳定。结果表明,\tool 可以极大地补充或增强现有的商业漏洞数据库。
    \item [(5)] \textbf{实用性:}在实际工作场景下,\tool 有助于用户更准确、更快速地识别到补丁。
\end{enumerate}


\section{展望}

\tool 以CVE ID作为输入,该方法不适用于没有CVE ID的开源漏洞,即没有收录在CVE平台中的漏洞。未来可以考虑以Advisory ID、Issue ID作为\tool 的输入,进一步挺高补丁覆盖率。

此外,\tool 中仅仅包含了四个知识源(NVD、Debian、Red Hat和GitHub),未来可以扩增更多的知识源(如,SecurityFocus\footnote{https://www.securityfocus.com}、CERT\footnote{https://www.kb.cert.org/vuls/}等),以构建更完整的漏洞参考链接网络。

当前,\tool 使用基于置信度和连通度的方法从参考链接网络中选择补丁,未来可以尝试使用自然语言和程序分析等技术,以实现基于语义的补丁识别方法。