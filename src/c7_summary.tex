\chapter{总结与展望}

% 本章节总结与展望。

\section{本文总结}
本文首先从补丁覆盖率、补丁一致性、补丁类型、漏洞与补丁之间的映射关系以及补丁准确性五个方面进行了一项经验研究,以了解两个商业漏洞数据库中开源软件漏洞补丁的质量和特征。

受经验研究结果的启发,本文提出一种名为\tool 的自动化补丁查找方法,以从多个知识源中查找开源软件漏洞的补丁。本文还设计了大量实验验证了\tool 的准确性、通用性、实用性等多个方面。\tool 的源代码和所有实验数据已经在\url{https://patch-tracer.github.io}网站上发布。

\section{未来展望}

\tool 以CVE-ID作为输入,该方法不适用于没有CVE-ID的开源漏洞,即没有收录在CVE平台中的漏洞。未来可以考虑以Advisory-ID、Issue-ID作为\tool 的输入。

此外,\tool 中仅仅包含了四个知识源,未来可以扩增更多的知识源(例如,SecurityFocus\footnote{https://www.securityfocus.com}),以构建更完整的漏洞信息网络。

当前,\tool 使用基于置信度和连通度的启发式方法从信息网络中选择补丁,未来可以尝试使用自然语言和程序分析等技术,以实现基于语义的补丁识别方法。