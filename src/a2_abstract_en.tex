% !TeX root = ../main.tex
\begin{abstract*}
Open source software (OSS) vulnerability management has become an open problem. Vulnerability databases provide valuable data that is needed to address OSS vulnerabilities. However, there arises a growing concern about the information quality of vulnerability databases. In particular, it is unclear how the quality of patches in existing vulnerability databases is. Further, existing manual or heuristic-based approaches for patch identification are either too expensive or too specific to be applied to all OSS vulnerabilities.

To address these problems, we first conduct an empirical study to understand the quality and characteristics of patches for OSS vulnerabilities in two state-of-the-art vulnerability databases. Our study is designed to cover five dimensions, i.e., the coverage, consistency, type, cardinality and accuracy of patches. Then, inspired by our study, we propose the first automated approach, named \tool, to find patches for an OSS vulnerability from multiple sources. Our key idea is that patch commits will be frequently referenced during the reporting, discussion and resolution of an OSS vulnerability.

Our extensive evaluation has indicated that i) \tool finds patches for up to \tocheck{273.8\%} more CVEs than existing heuristic-based approaches while achieving a significantly higher F1-score by up to \tocheck{116.8\%}; and ii) \tool achieves a higher recall by up to \tocheck{18.4\%} than state-of-the-art vulnerability databases, but sacrifices up to \tocheck{12.0\%} fewer CVEs (whose patches are not found) and \tocheck{6.4\%} lower precision. Our evaluation has also demonstrated the generality and usefulness of \tool.
\end{abstract*}