% !TeX root = ../main.tex
\begin{abstract*}
Open source software (OSS) vulnerability management has become an open problem of great concern.% Vulnerability databases are valuable by providing valuable data that is needed to address OSS vulnerabilities. As a result, 
OSS vulnerability databases, as the basis of security-related tasks, have arisen a growing concern about their information quality. However, it is unclear how the quality and features of patches in existing vulnerability databases are. Further, existing manual or heuristic-based approaches for patch identification are either too expensive or too specific to be applied to all OSS vulnerabilities.

To address these problems, we first conduct an empirical study to understand the quality and characteristics of patches for OSS vulnerabilities in two commercial vulnerability databases. Our study is designed to cover five dimensions, i.e., the coverage, consistency, type, cardinality and accuracy of patches. The result shows that (1) the quality of vulnerability patches in commercial vulnerability databases is not great. The lack of patches is prevalent, and the coverage of patches for vulnerabilities is only about 41.0\%. Patches in commercial vulnerability databases are of high accuracy, but for vulnerabilities with multiple patches, patches are often not complete. (2) In terms of types and mapping relationships, 93.7\% of patches are in the form of GitHub commits, and more than 40\% of vulnerabilities have a one-to-many mapping relationship with patches.

Then, inspired by our study, we propose an automated approach, named \tool, to identify patches for an OSS vulnerability from multiple sources.\tool is designed to identify patches of commit types and build a one-to-many mapping between vulnerabilities and patches. Our key idea is that patch commits will be frequently referenced during the reporting, discussion and resolution of an OSS vulnerability. Therefore, we first design a reference network based on multiple sources, and then select patch nodes with the highest confidence and connectivity from the network. Based on selected patches, \tool expands the patch list for vulnerabilities.

With five research questions, \tool is evaluated in terms of accuracy, generality, usefulness, etc. The extensive evaluation has indicated that (1) on the dataset of 1,295 CVEs, \tool can achieve the coverage, precision, and recall in 88.0\%, 0.86, and 0.864 respectively. (2) Compared with existing heuristic-based approaches, \tool improves the patch coverage by 58.6\% to 273.8\%, and the F1 value by 116.8\%. (3) Compared with commercial vulnerability databases $DB_A$ and $DB_B$, \tool improves the recall by 18.4\%, but decreases the coverage and precision by 12.0\% and 6.4\% respectively. This suggests that \tool can be used to supplement the patch missing from commercial vulnerability databases. (4) \tool is general to a wider range of OSS vulnerabilities. \tool can still identify patches for a larger number of vulnerabilities, which are patchless in $DB_A$ and $DB_B$. This suggests that \tool can greatly enhance existing commercial vulnerability databases. In practice, \tool assists users in identifying patches more accurately and quickly.
\end{abstract*}