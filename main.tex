%*********************************************************************
% fduthesis: 复旦大学论文模板
% 2020/08/30 v0.7e
%
% 重要提示:
%   1. 请确保使用 UTF-8 编码保存
%   2. 请使用 XeLaTeX 或 LuaLaTeX 编译
%   3. 请仔细阅读用户文档
%   4. 修改、使用、发布本文档请务必遵循 LaTeX Project Public License
%   5. 不需要的注释可以尽情删除
%*********************************************************************

\documentclass[type=master, oneside, draft=false]{fduthesis}
% 模板选项:
%   type = doctor|master|bachelor  论文类型,默认为本科论文
%   oneside|twoside                论文的单双面模式,默认为 twoside
%   draft = true|false             是否开启草稿模式,默认关闭
% 带选项的用法示例:
%   \documentclass[oneside]{fduthesis}
%   \documentclass[twoside, draft=true]{fduthesis}
%   \documentclass[type=bachelor, twoside, draft=true]{fduthesis}

% 页眉:左标题右章节
% \makeatletter
% \let\ps@plain\ps@fancy
% \makeatother
% \fancyhf{}
% \fancyhead[L]{\small\nouppercase{基于多源知识的安全漏洞影响组件的识别方法}}
% \fancyhead[R]{\small\nouppercase{\csname l__fdu_header_center_mark_tl\endcsname\leftmark}}
% \fancyfoot[C]{\small\thepage}
% \pagestyle{fancy}

% 魔改模板:根据学院要求改的格式
\makeatletter
\ExplSyntaxOn
% 专业:改为“专业学位类别(领域)”
% \tl_set:Nn \c__fdu_name_major_tl { 专业学位类别(领域) }
\__fdu_patch_cmd:Nnn \__fdu_cover_info: { 6em } { 9em }
% 页眉:左论文标题、右章节名
\let\ps@plain\ps@fancy
\fancyhf{}
\fancyhead[L]{\small\nouppercase{\fdu@kai\l__fdu_info_title_tl}}
\fancyhead[R]{\small\nouppercase{\fdu@kai\l__fdu_header_center_mark_tl\leftmark}}
\fancyfoot[C]{\small\thepage}
\pagestyle{fancy}
% 摘要:用逗号而不是封号分隔关键词
% \cs_set_protected:Npn \__fdu_abstract_end:
% {
%   \__fdu_keywords:nNn
%   { \sffamily \c__fdu_name_keywords_tl \c__fdu_fwid_colon_tl }
%   \l__fdu_info_keywords_clist { \c__fdu_fwid_comma_tl }
%   \__fdu_clc:nn
%   { \sffamily \c__fdu_name_clc_tl \c__fdu_fwid_colon_tl }
%   { \l__fdu_info_clc_tl }
% }
% \cs_set_protected:Npn \__fdu_abstract_en_end:
% {
%   \__fdu_keywords:nNn
%   { \bfseries \c__fdu_name_keywords_en_tl \__fdu_quad: }
%   \l__fdu_info_keywords_en_clist { , ~ }
%   \__fdu_clc:nn
%   { \bfseries \c__fdu_name_clc_en_tl \__fdu_quad: } 
%   { \l__fdu_info_clc_tl } 
% }
% 摘要:页数从罗马小写改为大写
\ctex_patch_cmd:Nnn \frontmatter
  { roman } { Roman }
\ExplSyntaxOff
\makeatother

\fdusetup{
  % 参数设置
  % 允许采用两种方式设置选项:
  %   1. style/... = ...
  %   2. style = { ... = ... }
  % 注意事项:
  %   1. 不要出现空行
  %   2. “=” 两侧的空格会被忽略
  %   3. “/” 两侧的空格不会被忽略
  %   4. 请使用英文逗号 “,” 分隔选项
  %
  % style 类用于设置论文格式
  style = {
    font = times,
    % 西文字体(包括数学字体)
    % 允许选项:
    %   font = garamond|libertinus|lm|palatino|times|times*|none
    %
    cjk-font = windows,
    % 中文字体
    % 允许选项:
    %   cjk-font = adobe|fandol|founder|mac|sinotype|sourcehan|windows|none
    %
    % 注意:
    %   1. 中文字体设置高度依赖于系统。各系统建议方案:
    %        windows:cjk-font = windows
    %        mac:    cjk-font = mac
    %        linux:  cjk-font = fandol(默认值)
    %   2. 除 fandol 和 sourcehan 外,其余字体均为商用字体,请注意版权问题
    %   3. 但 fandol 字体缺字比较严重,而 sourcehan 没有配备楷体和仿宋体
    %   4. 这里中西文字体设置均注释掉了,即使用默认设置:
    %        font     = times
    %        cjk-font = fandol
    %   5. 使用 font = none / cjk-font = none 关闭默认字体设置,需手动进行配置
    %
    font-size = -4,
    % 字号
    % 允许选项:
    %   font-size = -4|5
    %
    % fullwidth-stop = catcode,
    % 是否把全角实心句点 “.” 作为默认的句号形状
    % 允许选项:
    %   fullwidth-stop = catcode|mapping|false
    % 说明:
    %   catcode   显式的 “。” 会被替换为 “.”(e.g. 不包括用宏定义保存的 “。”)
    %   mapping   所有的 “。” 会被替换为 “.”(使用 LuaLaTeX 编译则无效)
    %   false     不进行替换
    %
    footnote-style = xits,
    % 脚注编号样式
    % 允许选项:
    %   footnote-style = plain|libertinus|libertinus*|libertinus-sans|
    %                    pifont|pifont*|pifont-sans|pifont-sans*|
    %                    xits|xits-sans|xits-sans*
    %
    hyperlink = none,
    % 超链接样式
    % 允许选项:
    %   hyperlink = border|color|none
    %
    % hyperlink-color = default,
    % 超链接颜色
    % 允许选项:
    %   hyperlink-color = default|classic|elegant|fantasy|material|
    %                     business|science|summer|autumn|graylevel|prl
    % 默认与西文字体保持一致
    %
    bib-backend = bibtex,
    % 参考文献支持方式
    % 允许选项:
    %   bib-backend = bibtex|biblatex
    %
    % bib-style = numerical,
    % 直接TMD改模板好了:D:\texlive\2021\texmf-dist\bibtex\bst\gbt7714\gbt7714-numerical.bst
    % Ban: address, publisher, doi, 非 EB/OL 的 url
    % 参考文献样式
    % 允许选项:
    %   bib-style = author-year|numerical|<其他样式>
    % 说明:
    %   author-year  著者—出版年制
    %   numerical    顺序编码制
    %   <其他样式>   使用其他 .bst(bibtex)或 .bbx(biblatex)格式文件
    %
    % cite-style = {},
    % 引用样式
    % 默认为空,即与参考文献样式保持一致
    % 仅适用于 biblatex;如要填写,需保证相应的 .cbx 格式文件能被调用
    %
    bib-resource = {res/ref.bib},
    % 参考文献数据源
    % 可以是单个文件,也可以是用英文逗号 “,” 隔开的一组文件
    % 如果使用 biblatex,则必须明确给出 .bib 后缀名
    %
    % logo = {fudan-name.pdf},
    % 封面中的校名图片
    % 模版已自带,通常不需要额外配置
    %
    % logo-size = {0.5\textwidth},      % 只设置宽度
    % logo-size = {{}, 3cm},            % 只设置高度
    % logo-size = {8cm, 3cm},           % 设置宽度和高度
    % 设置校名图片的大小
    % 通常不需要调整
    %
    auto-make-cover = false % 盲审要求
    % 是否自动生成论文封面(封一)、指导小组成员名单(封二)和声明页(封三)
    % 除非特殊需要(e.g. 不要封面),否则不建议设为 false
  },
  %
  % info 类用于录入论文信息
  info = {
    title = {基于多源信息的开源软件漏洞的补丁识别方法},
    % 中文标题
    % 长标题建议使用 “\\” 命令手动换行(不是指在源文件里输入回车符,当然
    % 源文件里适当的换行可以有助于代码清晰):
    %   title = {最高人民法院、最高人民检察院关于适用\\
    %            犯罪嫌疑人、被告人逃匿、死亡案件违法所得\\
    %            没收程序若干问题的规定},
    %
    title* = {Finding Patches for Open Source Software Vulnerabiliies from multiple sources},
    % 英文标题
    %
    author = {许聪颖},
    % author = {},  % 盲审要求
    % 作者姓名
    %
    % author* = {Your name},
    % 作者姓名(英文 / 拼音)
    % 目前不需要填写
    %
    supervisor = {陈碧欢\quad 副教授},
    % supervisor = {},  % 盲审要求
    % 导师
    % 姓名与职称之间可以用 \quad 打印一个空格
    %
    major = {软件工程},
    % 专业
    %
    degree = academic,
    % 学位类型
    % 允许选项:
    %   degree = academic|professional
    % 说明:
    %   academic      学术学位
    %   professional  专业学位
    %
    department = {软件学院},
    % 院系
    %
    student-id = {19212010035},
    % student-id = {},  % 盲审要求
    % 作者学号
    %
    % date = {2020 年 1 月 1 日},
    % 日期
    % 注释掉表示使用编译日期
    %
    % secret-level = ii,
    % 密级
    % 允许选项:
    %   secret-level = none|i|ii|iii
    % 说明:
    %   none  不显示密级与保密年限
    %   i     秘密
    %   ii    机密
    %   iii   绝密
    %
    % secret-year = {五年},
    % 保密年限
    % secret-level = none 时该选项无效
    %
    instructors = {
      {赵文耘 \quad 教\quad 授},
      {彭\quad 鑫 \quad 教\quad 授},
      {陈碧欢     \quad 副教授},
      {沈立炜     \quad 副教授}
    },
    % instructors = {}, % 盲审要求
    % 指导小组成员
    % 使用英文逗号 “,” 分隔
    % 如有需要,可以用 \quad 手工对齐
    %
    keywords = {关键词1, 关键词2, 关键词3},
    % 中文关键字
    % 使用英文逗号 “,” 分隔
    %
    keywords* = {Keyword1, Keyword2, Keyword3},
    % 英文关键字
    % 使用英文逗号 “,” 分隔
    %
    clc = {TP311}
    % 中图分类号
  }
}

% 需要的宏包可以自行调用
% \usepackage{amsthm}
\usepackage{amsmath,bm}
% \theoremstyle{acmdefinition}
\newtheorem{exmp}{Example} %这个地方贼麻烦,解决了老半天
\usepackage{booktabs}
\usepackage[labelsep=quad]{caption}
\usepackage{diagbox}
\usepackage{enumerate}
\usepackage{enumitem}
\usepackage{listings}
\usepackage{makecell}
\usepackage{physics}
\usepackage{threeparttable}
\usepackage{xcolor}
\usepackage{xspace}
\usepackage{tcolorbox}
\usepackage{stfloats}
\usepackage{subcaption}
\usepackage{graphicx}
\usepackage{multirow}
\usepackage{url}




\setenumerate[1]{itemsep=0pt,partopsep=0pt,parsep=\parskip,topsep=5pt}
\setitemize[1]{itemsep=0pt,partopsep=0pt,parsep=\parskip,topsep=5pt}
\setdescription{itemsep=0pt,partopsep=0pt,parsep=\parskip,topsep=5pt}
\ctexset{subsection/tocline={\CTEXnumberline{#1}#2}}

\lstset{
  % basicstyle=\small\tt,
  basicstyle=\normalfont\ttfamily,
  numbers=left,
  % numberstyle=\small\tt,
  numberstyle=\normalfont\ttfamily\color{darkgray},
  stepnumber=1,
  numbersep=8pt,
  frame=single,
  xleftmargin=3em,
  xrightmargin=1em,
  framexleftmargin=2em,
  captionpos=b,
  keywordstyle=\color{blue!70}, commentstyle=\color{red!50!green!50!blue!50},
  rulesepcolor=\color{red!20!green!20!blue!20},
}
\lstdefinelanguage{json}{
  literate=
    {\{}{{{\color{blue}{\{}}}}{1}
    {\}}{{{\color{blue}{\}}}}}{1}
    {[}{{{\color{blue}{[}}}}{1}
    {]}{{{\color{blue}{]}}}}{1},
}

% 需要的命令可以自行定义
\newcommand{\hilbertH}{\symcal{H}}
\newcommand{\ee}{\symrm{e}}
\newcommand{\ii}{\symrm{i}}
\newcommand{\ds}{\,\cdot\,}

\newcolumntype{C}[1]{>{\centering\arraybackslash}m{#1}}

\makeatletter
\newcommand\footnoteref[1]{\protected@xdef\@thefnmark{\ref{#1}}\@footnotemark}
\makeatother


\includeonly{
  src/a1_abstract.tex, src/a2_abstract_en.tex,
  src/c1_introduction.tex, src/c2_background.tex, src/c3_empirical.tex,
  src/c4_approach.tex, src/c5_experiment.tex, src/c6_related.tex, src/c7_summary.tex,
  src/z1_acknowledgement.tex, src/z2_review.tex
}

\begin{document}
% 盲审要求:指导小组成员不要了,独创性声明不要了
\begin{titlepage}
  \makecoveri     % 封面
  \newpage
  % \makecoverii    % 指导小组
\end{titlepage}

\renewcommand{\captionfont}{\small}
\renewcommand{\lstlistingname}{代码}
\renewcommand{\lstlistlistingname}{代码}
\renewcommand{\thelstlisting}{\thechapter-\arabic{lstlisting}}

\newcommand{\congyingEdit}[1]{\textcolor{blue}{#1}}
\newcommand{\tocheck}[1]{\textcolor{red}{#1}}
\newcommand{\tool}{\textsc{Tracer}\xspace}
\newcommand{\fn}[1]{\footnote{\scalebox{0.82}{#1}}}

% 这个命令用来关闭版心底部强制对齐,可以减少不必要的 underfull \vbox 提示,但会影响排版效果
% \raggedbottom

% 前置部分包含目录、中英文摘要以及符号表等
\frontmatter

\tableofcontents                  % 目录
% !TeX root = ../main.tex
\begin{abstract}

开源软件 (Open source software, OSS) 漏洞管理已然成为一个热点问题。开源漏洞数据库为解决漏洞问题提供十分有价值的数据信息,因此,漏洞数据库的数据质量也受到越来越多的关注和研究。具体的问题为:现有漏洞数据库中补丁的质量尚未研究清楚,此外,现有的补丁信息多由人工或基于启发式的识别方法进行收集。这种方法人工成本过高,且过于定制化无法应用于全部的OSS漏洞。

\congyingEdit{empirical study该如何翻译比价好呢?实证研究?经验性研究?}
为了解决这些问题,首先,我们进行了实证研究,以了解当前商业旗舰漏洞数据库中开源软件漏洞补丁的质量和特征。我们的研究涵盖五个方面,包括:补丁的覆盖度、一致性、类型、\congyingEdit{基数}和准确性。然后,基于研究的发现,我们提出了第一种名为 \tool 的自动化方法,用于从多个来源查找开源漏洞的补丁。%我们的主要思想是在报告、讨论和解决 OSS 漏洞期间会经常引用补丁提交。

实验评估表明:i) 与现有的基于启发式的方法相比,\tool 能够为多达 \tocheck{273.8\%} 的 CVE找到补丁;同时,准确性方面,将F1数值提高达 \tocheck{116.8\%};ii) 与现有的漏洞数据库相比,\tool 将召回率(recall)提高达 \tocheck{18.4\%} ;然而,\tocheck{12.0\%} 的 CVE补丁未找到,精度(precision)下降约\tocheck{6.4\%}。%我们的评估也证明了\tool 的通用性和实用性。

\end{abstract}
     % 摘要
% !TeX root = ../main.tex
\begin{abstract*}
Open source software (OSS) vulnerability management has become an open problem. Vulnerability databases provide valuable data that is needed to address OSS vulnerabilities. However, there arises a growing concern about the information quality of vulnerability databases. In particular, it is unclear how the quality of patches in existing vulnerability databases is. Further, existing manual or heuristic-based approaches for patch identification are either too expensive or too specific to be applied to all OSS vulnerabilities.

To address these problems, we first conduct an empirical study to understand the quality and characteristics of patches for OSS vulnerabilities in two state-of-the-art vulnerability databases. Our study is designed to cover five dimensions, i.e., the coverage, consistency, type, cardinality and accuracy of patches. Then, inspired by our study, we propose the first automated approach, named \tool, to find patches for an OSS vulnerability from multiple sources. Our key idea is that patch commits will be frequently referenced during the reporting, discussion and resolution of an OSS vulnerability.

Our extensive evaluation has indicated that i) \tool finds patches for up to \tocheck{273.8\%} more CVEs than existing heuristic-based approaches while achieving a significantly higher F1-score by up to \tocheck{116.8\%}; and ii) \tool achieves a higher recall by up to \tocheck{18.4\%} than state-of-the-art vulnerability databases, but sacrifices up to \tocheck{12.0\%} fewer CVEs (whose patches are not found) and \tocheck{6.4\%} lower precision. Our evaluation has also demonstrated the generality and usefulness of \tool.
\end{abstract*}  % 英文摘要

% 主体部分是论文的核心
\mainmatter

\chapter{绪论}

% 本章节概述了背景、研究目的与意义\cite{jia2021:oss-vulnerability, mitre2021:cve}。
本章将阐述本文的研究背景、研究问题、主要工作、主要贡献以及本文的篇章结构。

\section{研究背景}

开源软件(Open Source Software,OSS)为开源及闭源应用程序的快速开发提供了基础。得益于开源社区的蓬勃发展,在软件开发过程中,开发人员经常会使用开源软件中已实现的功能,节省开发时间,加快开发速度\cite{Wang2020empirical}。然而,伴随着开发效率的提高,开源软件中的安全漏洞也会被引入软件系统\cite{2何熙巽2020软件供应链安全综述,3刘剑2018软件与网络安全研究综述}。据Synopsys公司发布的《开源安全和风险分析报告》\footnote{https://www.synopsys.com/content/dam/synopsys/sig-assets/reports/rep-ossra-2021.pdf}显示,在该公司分析的1,500个应用程序中,98\%的应用程序都使用了开源软件。
%然而,大规模使用开源软件可以加速应用程序开发的进程,但同时也引入了安全风险。
报告还指出,高达84\%的应用程序包含至少一个已知的开源软件漏洞,对比于前一年(2019年)增加了9\%。此外,据Snyk公司发布的报告\footnote{https://snyk.io/wp-content/uploads/sooss\_report\_v2.pdf}显示,近些年开源软件中所披露的漏洞越来越多,过去两年间几乎翻了一倍。

针对以上问题,大量工作都在研究如何降低开源软件漏洞带来的安全风险,包括通过学习漏洞特征来检测开源软件中的漏洞\cite{li2016vulpecker,li2018vuldeepecker,zhou2019devign,jimenez2019importance}、通过匹配漏洞及补丁签名来检测开源软件漏洞\cite{jang2012redebug, kim2017vuddy, xu2020patch, xiao2020mvp, cui2020vuldetector}%、修复开源软件中的漏洞\cite{mulliner2013patchdroid, duan2019automating, xu2020automatic, machiry2020spider}、
以及进行软件成分分析以确定应用程序中的开源软件漏洞是否被执行\cite{pashchenko2018vulnerable, ponta2020detection, pashchenko2020vuln4real, Wang2020empirical}。在这些工作中,准确且完整的漏洞知识十分重要。例如,漏洞描述、受漏洞影响的软件、版本以及补丁等知识都是这些工作得以开展的基础。目前,已有多方人员致力于构建安全漏洞数据库。在安全社区中,由美国政府资助的CVE List\footnote{https://cve.mitre.org/cve/}(Common Vulnerabilities \& Exposures)、NVD\footnote{https://nvd.nist.gov}(National Vulnerability Database)和由中国政府资助的CNVD\footnote{https://www.cnvd.org.cn/}(国家信息安全漏洞共享平台)、CNNVD\footnote{http://www.cnnvd.org.cn/}(国家信息安全漏洞数据库)是最具影响力的漏洞数据库。在工业界中,BlackDuck\footnote{https://www.synopsys.com/content/dam/synopsys/sig-assets/datasheets/bdknowledgebase-ds-ul.pdf}、WhiteSource\footnote{https://www.whitesourcesoftware.com/vulnerability-database/}、Veracode\footnote{https://sca.veracode.com/vulnerability-database/search}和Snyk\footnote{https://snyk.io/vuln}等公司较为关注开源软件中的安全漏洞,并已经构建各自的商业漏洞数据库,作为安全服务的基础。在学术界中,也有很多工作致力于构建漏洞数据集\cite{ponta2019manually,fan2020ac,jimenez2018enabling,gkortzis2018vulinoss,namrud2019androvul},但这些数据集大多是针对特定语言的生态系统或针对特定的软件项目而设计。

\section{研究问题}
% \textbf{研究问题:}
随着构建的漏洞数据库越来越多,数据库中积累的漏洞数据也越来越多,研究人员开始关注数据库中漏洞知识的质量。Dong等人\cite{dong2019towards}发现了漏洞数据库中受漏洞影响的软件版本信息不准确的情况,Chaparro等人\cite{chaparro2017detecting}和Mu等人\cite{mu2018understanding}发现了漏洞描述中普遍缺失关键的漏洞重现步骤。这种信息不完整或不准确的情况使得安全工作人员难以及时地识别、重现和修复应用程序中的漏洞。

漏洞补丁作为刻画漏洞特征的重要知识,可应用于多种安全相关的任务,包括补丁生成和热部署\cite{mulliner2013patchdroid,duan2019automating,xu2020automatic}、补丁存在测试\cite{zhang2018precise,jiang2020pdiff,dai2020bscout}、软件成分分析\cite{ponta2020detection,pashchenko2020vuln4real,Wang2020empirical}、漏洞检测\cite{li2016vulpecker,li2018vuldeepecker,jang2012redebug,kim2017vuddy, xiao2020mvp, cui2020vuldetector}等。如果漏洞数据库中的补丁知识缺失或不准确,那么这些安全任务的准确性将会受到严重影响。然而,漏洞数据库中的补丁知识尚未被系统地研究和评估,目前尚不清楚现有漏洞数据库中补丁的质量情况。

此外,现有的漏洞补丁识别方法主要有三种:(1)人工手动查找漏洞补丁\cite{xu2020automatic,jiang2020pdiff,dai2020bscout,zhou2017automated,sabetta2018practical,chen2020machine,xiao2020mvp,ponta2020detection,pashchenko2020vuln4real}。(2)通过启发式规则识别漏洞补丁,比如在NVD参考链接中查找代码提交\cite{duan2019automating,li2016vulpecker}或是在代码仓的提交历史中搜索漏洞标识符(CVE ID)\cite{you2017semfuzz,Wang2020empirical}。(3)在特定项目的安全公告中搜索漏洞补丁\cite{mulliner2013patchdroid,jang2012redebug,kim2017vuddy}。以上这些方法的人工成本过高,且针对特定的程序语言或项目设计,无法广泛应用于所有开源软件漏洞。

综上,目前的问题是,漏洞数据库中补丁的特征及质量尚未被系统地研究和评估,并且已有的漏洞补丁采集方法通用性较差且人工成本过高。

\section{本文工作}
为解决上述研究问题,本文先开展了一项针对开源软件漏洞补丁的经验研究,以了解当前商业漏洞数据库中开源软件漏洞补丁的质量和特征。然后,基于经验研究的发现,本文提出了一种名为\tool 的基于多源知识的开源软件漏洞的补丁识别方法。该方法通过构建漏洞的多源参考链接网络来识别补丁。本文还进行了大量实验,从准确性、通用性、实用性等多个方面对\tool 进行了评估。

\subsection{开源软件漏洞补丁的经验研究}
为了了解当前商业漏洞数据库中开源软件漏洞补丁的质量和特征,本文挑选了两个认可度较高的商业漏洞数据库作为研究对象。该经验研究涵盖五个方面,包括补丁覆盖度分析、补丁一致性分析、补丁类型分析、补丁映射关系以及补丁准确性分析。

本文首先构建了一个广度数据集,该数据集包含10,070个开源软件漏洞。在此数据集上,本文分析两个商业漏洞数据库中开源软件漏洞补丁的覆盖率和一致性。结果表明:在漏洞补丁覆盖率方面,\tocheck{10,070}个漏洞中只有\tocheck{4,602(5.7\%)}的漏洞在商业漏洞数据库中提供了补丁;在漏洞补丁一致性方面,只有\tocheck{19.7\%}的漏洞在两个商业漏洞数据库中有一致的补丁。%\congyingEdit{可以再写些其他的结果}。

基于广度数据集中含补丁的漏洞数据,本文还通过人工构建了一个深度数据集。该数据集包含1,295个开源软件漏洞,且这些漏洞都有补丁。在此数据集上,本文分析了开源软件漏洞补丁的类型和漏洞补丁的映射关系,并评估了商业数据库中漏洞补丁的准确性。结果表明:在漏洞补丁类型方面,\tocheck{1,265(97.7\%)}漏洞的补丁类型都是GitHub或SVN的代码提交;在漏洞补丁映射关系方面,\tocheck{533(41.1\%)}的漏洞与其补丁有一对多的映射关系;在漏洞补丁准确性方面,两个商业数据库的补丁精确率都高于\tocheck{90\%},但对于一对多映射类型的漏洞,两个商业数据库中补丁的召回率都仅为\tocheck{50\%}左右。

这些结果表明,现有的商业漏洞数据库中缺失了许多漏洞的补丁,尤其是对于有多个补丁的漏洞,补丁缺失现象更为严重。这种信息不完整或不准确的情况,使得安全工作人员难以及时地识别、重现和修复开源软件中的漏洞。同时,这也反映出自动化的补丁识别方法的需求,自动化的方法可以帮助工作人员识别并补全缺失的补丁。

\subsection{开源软件漏洞补丁的识别方法}
基于经验研究的发现,本文提出了一种名为\tool 的基于多源知识的开源软件漏洞的补丁识别方法。该方法从多个知识源(即NVD\footnote{https://nvd.nist.gov}、Debian\footnote{https://security-tracker.debian.org/tracker/}、Red Hat\footnote{https://bugzilla.redhat.com/}以及GitHub\footnote{https://github.com/})构建漏洞的参考链接网络并识别补丁(参考链接,即URL网址)。该方法的核心思想是:漏洞的补丁链接会在与该漏洞相关的各种来源的漏洞公告、分析报告、讨论和解决的过程中被频繁提及和引用。因此,本文首先设计了一种基于多知识源的漏洞参考链接网络,然后再从该网络中选出具有最高置信度和连通度的补丁节点作为结果,并基于选定的补丁进行补丁扩增,从而构建一对多的漏洞补丁映射关系。

\tool 以漏洞的CVE ID作为输入,经过三个步骤,输出该漏洞的补丁。(1)多源参考链接网络构建,该步骤的目的是将该漏洞在被报告、讨论和解决阶段的参考链接进行建模。\tool 从多个\tocheck{漏洞知识源}(即NVD、Debian、Red Hat和GitHub)中提取引用的参考链接信息并构建一个参考链接网络。(2)补丁选择,\tool 从构建的参考链接网络中选择中具有高连通性和高置信度的补丁节点作为该漏洞的补丁。(3)补丁扩增,基于前一步骤选定的补丁,\tool 通过搜索同一代码库所有分支中的相关提交来扩展补丁集,构建一对多的漏洞补丁映射关系。最终,返回所有选中及扩展的补丁。

\subsection{实验评估}
本文进行了大量实验,通过五个研究问题,从准确性、通用性、实用性等多个方面对\tool 进行了评估。为了评估\tool 的准确性,本文将\tool 与三种基于启发式规则的方法以及两个商业漏洞数据库进行了比较。结果表明,在包含1,295个漏洞的深度数据集上,
(1)\tool 可以达到88.0\%的补丁覆盖率、0.864的补丁精确率和0.864的补丁召回率;
(2)与现有的基于启发式规则的方法相比,\tool 将补丁覆盖率提高\tocheck{58.6\%}到\tocheck{273.8\%};同时,在补丁的准确性上,\tool 的F1数值也比基于启发式规则的方法高116.8\%;
(3)与现有的商业漏洞数据库$DB_A$和$DB_B$相比,\tool 的补丁召回率高出18.4\%;但仍有155(12.0\%)的漏洞\tool 未能找到补丁,补丁精确率也低了6.4\%。%;(3)\tool 对于大范围开源软件漏洞具有较好的通用性,在实际使用中,也具有较好的实用性。

此外,为了评估\tool 的通用性,本文还另外构建了两个更大的、包含\tocheck{3,185} 和\tocheck{5,468}个漏洞的数据集,并在这两个数据集上运行\tool 。结果表明,\tool 在两个数据集上分别可以找到\tocheck{67.7\%}和\tocheck{51.5\%}个漏洞的补丁;补丁精确率分别为\tocheck{0.823} 和\tocheck{0.888},补丁召回率分别为\tocheck{0.845}和\tocheck{0.899},这表明\tool 在识别补丁方面具有较好的通用性。

此外,为了评估\tool 在实际工作种的实用性,本文还邀请了10名实验人员进行了用户研究。评估结果表明,在实际使用中,\tool 有助于用户更准确、更快速地识别到补丁。

\subsection{主要贡献}
本文主要有以下贡献:
\begin{enumerate}
\item [(1)]本文开展了一项针对开源软件漏洞补丁的经验研究,以了解当前商业漏洞数据库中开源软件漏洞补丁的质量和特征。该经验研究涵盖五个方面,包括补丁覆盖度分析、补丁一致性分析、补丁类型分析、补丁映射关系以及补丁准确性分析。
\item [(2)]本文提出了一种名为 \tool 的基于多源知识的开源软件漏洞的补丁识别方法,该方法可服务于安全社区、工业界和学术界的研究人员。
\item [(3)]本文进行了针对\tool 准确性、通用性、实用性等多个方面的实验评估。
\end{enumerate}


\section{本文篇章结构}
本文共包含六个章节,结构如下:

第一章绪论,介绍了本文的研究背景及研究问题,简述了本文的主要工作(包括经验研究、方法设计和实验评估)、主要贡献以及篇章结构。

第二章背景知识及相关工作,介绍了本文所涉及的背景知识,包括通用漏洞披露CVE、漏洞公告、漏洞补丁等信息,为后文经验研究、方法设计等内容的展开做铺垫。本章还介绍了与本文研究主题的相关工作,包括漏洞信息质量、漏洞补丁识别以及漏洞补丁应用。

第三章经验研究,介绍了本文为了解当前商业漏洞数据库中开源软件漏洞补丁的质量和特征,针对开源软件漏洞补丁所开展的的经验研究,涵盖补丁覆盖度分析、补丁一致性分析、补丁类型分析、补丁映射关系以及补丁准确性分析五个方面。

第四章\tool 方法设计,介绍了本文提出的一种名为\tool 的基于多源知识的开源软件漏洞的补丁识别方法。该方法包括多源参考链接网络构建、补丁选择以及补丁扩增三个步骤。

第五章实验评估,介绍了本文针对\tool 的准确性、通用性、实用性等五个方面所进行的实验评估。

第六章总结与展望,对本文的工作内容及研究成果进行总结,讨论本文研究工作中存在的不足及可以改进的地方,并展望了未来可以进行的工作。
     % 绪论
\chapter{背景知识及相关工作}
本文的研究重点是开源软件漏洞的补丁定位问题,本章将首先介绍漏洞相关的背景知识,包括:通用漏洞披露(CVE)、漏洞通告(advisory)和漏洞补丁(vulnerability patch);然后,从漏洞披露信息的质量、漏洞补丁的分析及应用三个方面介绍相关的研究工作。


\section{背景知识}
% 本小节主要介绍本文工作中的背景知识,包括:通用漏洞披露(CVE)、漏洞通告(advisory)和漏洞补丁(vulnerability patch)。

\subsection{CVE及NVD} 
% 重平台
% 是什么? 有哪些信息?举个例子? 
% 作为community的source,还有其他的third party db、sources?
% 有什么作用? 被哪些其他 official、third-party report引用
通用漏洞披露(Common Vulnerabilities and Exposures,CVE)\cite{mitre2021:cve},是一个与网络安全有关的漏洞字典,收集各种信息安全漏洞并给予唯一编号以便于公众查阅及引用\footnote{https://cve.mitre.org/cve/}。在实际使用中,当人们提及某个CVE时,他们其实是在说某个被分配了CVE-ID的安全漏洞\footnote{https://www.redhat.com/en/topics/security/what-is-cve}。
\begin{figure}[h]
    \centering
    \includegraphics[width=0.88\textwidth]{fig/CVE-2021-44228-2}
    \caption{CVE平台中漏洞CVE-2021-44228信息}
    \label{fig:CVE-2021-44228}
\end{figure}

每一个CVE条目(CVE Entry)都有唯一通用标识符(即:CVE-ID)、一段漏洞描述(即:Description)以及至少一个引用链接(即:References),该引用链接多为外部网站且包含与该漏洞相关的更详细的描述信息。如图例\ref{fig:CVE-2021-44228}所示为CVE-2021-44228\footnote{https://cve.mitre.org/cgi-bin/cvename.cgi?name=CVE-2021-44228},漏洞的描述信息为:“Apache Log4j2 2.0-beta9 through 2.12.1 and 2.13.0 through 2.15.0 JNDI features used in configuration, ...... Note that this vulnerability is specific to log4j-core and does not affect log4net, log4cxx, or other Apache Logging Services projects.”,且包含多个引用链接,如:“https://www.kb.cert.org/vuls/id/930724”等URL。

\begin{figure}[h]
    \centering
    \includegraphics[width=1.0\textwidth]{fig/NVD-2021-44228}
    \includegraphics[width=1.0\textwidth]{fig/NVD-2021-44228-2}
    \caption{NVD平台中漏洞CVE-2021-44228信息}
    \label{fig:NVD-2021-44228}
\end{figure}

CVE被设计作为漏洞字典用于收录并编号漏洞,这也导致CVE中漏洞信息过于精简,仅有一段漏洞描述和引用链接信息。基于CVE平台收录的漏洞条目信息,美国国家漏洞数据库(NVD)\footnote{https://nvd.nist.gov/}、中国国家信息安全漏洞库(CNNVD)\footnote{http://www.cnnvd.org.cn/}等漏洞数据库被构建,与CVE平台数据完全同步,并为每个漏洞条目(CVE Entry)提供更丰富的信息,如:影响的软件名及版本、修复信息、严重性评分、影响评级等。如图\ref{fig:NVD-2021-44228}所示为NVD平台中漏洞CVE-2021-44228\footnote{https://nvd.nist.gov/vuln/detail/CVE-2021-44228}信息。



\subsection{漏洞公告} 
% 是什么? 有哪些信息? 有哪些种(official、third-party)?举个例子?
漏洞公告(Advisory),也称漏洞通告,是由一般是由受漏洞影响的软件的厂商(Vendor)对外发布的安全漏洞警报,一般包含:漏洞触发描述、漏洞影响结果、漏洞软件名、软件版本等漏洞描述信息,有时也会包含漏洞发现者、漏洞修复记录、漏洞补丁等信息。

\begin{figure}[h]
    \centering
    \includegraphics[width=0.88\textwidth]{fig/Vendor-2021-44228}
    \includegraphics[width=0.88\textwidth]{fig/Vendor-2021-44228-2}
    \caption{Apache Log4j发布的漏洞公告(CVE-2021-44228)}
    \label{fig:Vendor-2021-44228}
\end{figure}

如图\ref{fig:Vendor-2021-44228}所示为厂商Apache发布的关于CVE-2021-44228的漏洞公告\footnote{https://logging.apache.org/log4j/2.x/security.html},其中包含:修复方式“Log4j 2.x mitigation ...... Upgrade to Log4j 2.3.1 (for Java 6), 2.12.3 (for Java 7), or 2.17.0 (for Java 8 and later)”、漏洞发现者“Credit:This issue was discovered by Chen Zhaojun of Alibaba Cloud Security Team.”、引用链接“https://issues.apache.org/jira/browse/LOG4J2-3198”等信息。


以上由厂商发布的漏洞公告,也常被称为:厂商公告(Vendor Advisory);但由于某一厂商只会维护与该厂商相关的软件漏洞通告,开源社区的开发人员难以一一关注所有厂商的公告信息。因此,出现了许多第三方平台收集并维护所有漏洞公告信息,被称为:第三方公告(Third-Party Advisory),例如:Red Hat\footnote{https://access.redhat.com/errata/}、Debian\footnote{https://www.debian.org/security/}、Gentoo\footnote{https://security.gentoo.org/}等。

% 官方或第三方?再顺路引出相关的sources,作为例子。

% Vender advisory, official advisory:CVE NVD,third-party advisory:Redhat debian 啥啥
% 是什么? 有哪些信息? 有哪些种(official、third-party:权威或非权威啥啥啥。。。社区、工业啥啥、、、)?举个例子?

\begin{figure}[h]
    \centering
    \includegraphics[width=0.88\textwidth]{fig/debian-2021-44228}
    \includegraphics[width=0.88\textwidth]{fig/debian-2021-44228-2}
    \caption{Debian平台中漏洞CVE-2021-44228信息}
    \label{fig:debian-2021-44228}
\end{figure}
如图\ref{fig:debian-2021-44228}所示为Debian平台中漏洞CVE-2021-44228信息\footnote{https://security-tracker.debian.org/tracker/CVE-2021-44228},其中还甚至包括了该漏洞的补丁提交信息\footnote{https://github.com/apache/logging-log4j2/commit/c77b3cb39312b83b053d23a2158b99ac7de44dd3}(图中“Note”中的URL引用链接),这在NVD以及Vendor Advisory中都未出现。


\subsection{漏洞补丁}
补丁(Patch),也称:补丁程序,是指对计算机程序进行的一组更改,旨在更新其功能或修复其缺陷。漏洞补丁(Vulnerability Patch)则指为修复程序中的安全漏洞所开发的补丁,通常以代码提交(Git和SVN Commit),或文本文件(.patch文件)等形式。

如图\ref{fig:debian-2021-44228}中的URL引用链接“https://github.com/apache/logging-log4j2/commit/\\c77b3cb39312b83b053d23a2158b99ac7de44dd3” (Github Commit),即为漏洞CVE-2021-44228的补丁,补丁内容见图\ref{fig:debian-2021-44228}。

\begin{figure}[h]
    \centering
    \includegraphics[width=0.88\textwidth]{fig/commit-2021-44228}
    \caption{漏洞CVE-2021-44228的Github Commit补丁}
    \label{fig:commit-2021-44228}
\end{figure}



\section{相关工作}
\subsection{漏洞信息质量}
漏洞数据库(例如:CVE和NVD)被广泛关注,其中的漏洞信息也被广泛参考使用。随着漏洞数据库积累的漏洞数据越来越多,研究人员也越来越关注其中的漏洞信息的质量。

Nguyen和Massacci\cite{nguyen2013reliability}最早揭示了NVD数据库中漏洞所影响的软件版本信息的不可靠性。为了提高该信息的可靠性,Nguyen\cite{nguyen2016automatic}和Dashevskyi等人\cite{dashevskyi2018screening}开发了工具以确定某一旧软件版本是否会受到新披露的漏洞的影响。他们认为:如果旧版本包含修复漏洞所更改的源代码行,则该版本被视为受漏洞影响。Dong等人\cite{dong2019towards}从漏洞的描述信息中识别受漏洞影响的软件名称和版本,并与漏洞报告所提供的软件名称和版本信息进行对比。他们发现漏洞数据库中会遗漏真正受漏洞影响的版本,也会错误地包含了不受漏洞影响的版本。Chen等人\cite{chen2020automated}识别受漏洞影响的开源库信息。Chaparro等人的工作\cite{chaparro2017detecting}检测漏洞描述中是否缺少用于重现漏洞的关键步骤或预期效果信息。Mu等人的工作\cite{mu2018understanding}揭示了漏洞报告丢失重现漏洞信息的普遍性。以上工作已侧重于漏洞信息的多个方面,而本文的工作重点则是研究漏洞的补丁信息,并尝试自动化地从不同来源漏洞报告的综合信息中定位漏洞补丁。
% \tocheck{jo等人}\cite{jo2020gapfinder}识别网络安全领域内的语义不一致。
% \tocheck{这些工作侧重于漏洞信息的不同方面。按照这个方向,我们的工作重点是漏洞补丁,并尝试从漏洞报告的综合来源中识别漏洞补丁。}

近期,Tan等人完成了一项与本文的研究问题相似的工作\cite{Tan2021locating}。他们使用深度学习排名算法对代码仓库中的提交(commit)历史进行排名,把排在首位的提交当作为漏洞的补丁提交。他们的工作包含两个假设:(1)CVE中,受漏洞影响的软件的代码仓库已知;(2)漏洞与其补丁提交在数量上是一对一的映射关系。然而,事实上,受漏洞影响的软件的代码仓库并不已知,而需人工识别;此外,漏洞与其补丁提交在数量上存在一对多的关系(Sec.\ref{sec:cardinality})。


\subsection{漏洞补丁分析}
当前,有多中补丁分析相关的任务可用于提高软件安全性,如:补丁的生成和部署\cite{mulliner2013patchdroid,duan2019automating,xu2020automatic}、补丁的存在性测试\cite{zhang2018precise,jiang2020pdiff,dai2020bscout}以及秘密补丁识别\cite{xu2017spain,zhou2017automated,sabetta2018practical,chen2020machine}。

此外,研究人员也已为Java\cite{ponta2019manually}、C/C++\cite{fan2020ac}以及特定开源项目\cite{jimenez2018enabling}构建安全补丁数据集。基于这些数据集,研究人员已开展实证研究以表征漏洞及其补丁\cite{zaman2011security,li2017large,liu2020large,antal2020exploring}。在这些工作中,补丁信息多由人工识别\cite{xu2020automatic,jiang2020pdiff,dai2020bscout,xu2017spain,zhou2017automated,sabetta2018practical,chen2020machine,ponta2019manually,zaman2011security},或通过启发式规则识别,例如:在CVE的引用链接中查找补丁提交信息\cite{duan2019automating,fan2020ac,jimenez2018enabling,li2017large,liu2020large},以及在提交信息(Commit Message)中搜索CVE标识符\cite{fan2020ac, jimenez2018enabling, antal2020exploring}。这些工作存在的问题为:通过人工收集成本过高,且耗时较长;然而,启发式规则的方法又不足以找到或是找全补丁。


\subsection{漏洞补丁应用}
漏洞补丁信息可被用于多种软件安全性任务。例如,基于漏洞补丁生成漏洞攻击程序\cite{brumley2008automatic,you2017semfuzz},通过软件成分分析以确定项目是否使用包含漏洞的第三方库,并判定该漏洞所影响的函数是否被调用\cite{pashchenko2018vulnerable,ponta2020detection,pashchenko2020vuln4real,Wang2020empirical},以及通过学习漏洞特征\cite{li2016vulpecker,li2018vuldeepecker,zhou2019devign,jimenez2019importance}、通过匹配漏洞签名\cite{jang2012redebug,kim2017vuddy}、通过匹配漏洞和补丁检测签名\cite{xu2020patch,xiao2020mvp,cui2020vuldetector}来检测程序中的漏洞。


与上一小结的补丁分析工作类似,这些工作中的CVE补丁主要通过人工识别\cite{pashchenko2018vulnerable,ponta2020detection,pashchenko2020vuln4real,xiao2020mvp}、基于启发式规则的方法\cite{li2016vulpecker,li2018vuldeepecker,you2017semfuzz,Wang2020empirical,jimenez2019importance}或直接取自为特定项目建立CVE和补丁之间映射关系的安全公告\cite{jang2012redebug,kim2017vuddy,xu2020patch}。但是,人工识别的成本过高,而且基于启发式规则的方法找到或是找全补丁。       % 背景知识与相关工作
\chapter{经验研究}

本章将主要介绍为了了解当前漏洞数据库现状,考察其中漏洞补丁的质量和特征所开展的经验研究工作,包括:经验研究的设计、数据准备以及经验研究的结果分析。


\section{研究设计}
\subsection{研究问题}
为了了解已有漏洞数据库中开源软件漏洞补丁的质量和特征,本文所开展的针对当前\tocheck{高质量}漏洞数据库的经验研究包含以下研究问题:

\begin{itemize}[leftmargin=*]
    \item \textbf{RQ1 覆盖率分析:}当前\tocheck{高质量}漏洞数据库中,漏洞补丁信息的覆盖度如何?即:有多少漏洞包含补丁信息?(Sec. \ref{sec:coverage})
    \item \textbf{RQ2 一致性分析:}不用漏洞库间,漏洞补丁信息的一致性如何?即:有多少漏洞在漏洞数据库中具有相同的补丁信息?(Sec. \ref{sec:consistency})
    \item \textbf{RQ3 补丁类型分析:}开源漏洞补丁的类型有哪些? (Sec. \ref{sec:type})
    \item \textbf{RQ4 补丁映射分析:}开源漏洞与其补丁在数量上的映射关系是怎样的? (Sec. \ref{sec:cardinality})
    \item \textbf{RQ5 补丁准确性分析:}当前\tocheck{高质量}漏洞数据库中,漏洞的补丁信息准确度如何? (Sec. \ref{sec:accuracy})
\end{itemize}
    
其中,RQ1可用来评估漏洞数据库中开源软件漏洞的补丁缺失程度,RQ2用来评估不同漏洞数据库中漏洞补丁的不一致程度,RQ3和RQ4用来表征常见的补丁类型以及开源漏洞及其补丁之间的映射关系,RQ5可用来评估不同漏洞数据库中漏洞补丁信息的准确性。总的来说,RQ1、RQ2和RQ5的结果旨在从不同的角度评估补丁质量,并挖掘出对自动化补丁\tocheck{识别}方法的需求;RQ3和RQ4旨在从不同角度捕捉开源软件漏洞补丁的特征,并为自动化补丁\tocheck{识别}方法的设计提供\tocheck{启发}。

\subsection{评估标准【todo】}
precision、recall。。。

\section{数据准备}\label{sec:preparation}
\subsection{漏洞数据库选择}
为挑选高质量的、具有代表性的漏洞数据库作为研究对象,本文前期调研了来自安全领域的社区、工业界和学术界的漏洞数据库。在该章节的经验研究工作中,本文首先排除了来自安全社区的数据库(例如,CVE List和NVD)。因为这两个数据库不提供结构化的补丁信息,而补丁链接多是隐藏在参考链接中;此外,CVE List和NVD数据库中不仅仅包含开源软件漏洞,还包括闭源软件、系统及硬件相关的漏洞。本文还排除了来自学术界的数据集\cite{ponta2019manually,fan2020ac,jimenez2018enabling,gkortzis2018vulinoss,namrud2019androvul,li2017large,liu2020large,antal2020exploring},这是因为这些数据集中的漏洞通常限定于特定的一两种程序语言(例如:Python、Java),而非面向所有开源软件,缺乏多样性不具有代表性;此外,由于长期缺乏维护,这些漏洞数据集缺失较新的漏洞数据。


对于工业界的数据库,本文首先关注到BlackDuck\cite{blackduck}、WhiteSource\cite{whitesource}、Veracode\cite{veracode}和Snyk\cite{snyk}四家安全公司提供软件成分分析(Software Composition Analysis)服务,这种服务通过识别并分析当前软件系统中使用的开源成本(即:第三方库),报告所使用的开源成分中的漏洞。因此,这四家公司需要先构建尽可能完整且包含详细漏洞信息的漏洞库作为服务基础,本文便首先将这四家公司的漏洞数据库作为研究对象。截至2021年4月5日,收集的信息为:

\begin{itemize}[leftmargin=*]
\item\textbf{Black Duck,}该公司的报告显示:该公司的安全公告中共包含157,000多个漏洞,涵盖90多种编程语言,其中,数千个漏洞尚未被NVD收录。该公司的漏洞数据库由特定的专家团队进行维护,以确保漏洞数据的完整性和准确性,然而,该公司的漏洞数据信息并未对外公开。
\item\textbf{Sonatype\footnote{https://ossindex.sonatype.org},} 该公司声称:\textit{“OSS Index是一个免费的开源组件目录,其中的扫描工具可帮助开发人员识别漏洞、了解风险并确保其软件安全。”}\footnote{英文原文为:"OSS Index is a free catalogue of open source components and scanning tools to help developers identify vulnerabilities, understand risk, and keep their software safe."} Sonatype的OSS Index支持20多个生态系统(如:Maven、npm、Go、PyPI等)。该公司公开的漏洞信息包括:漏洞描述、受漏洞影响的组件和版本、CVSS向量和参考链接等信息。
\item\textbf{WhiteSource\tocheck{链接},}该公司从NVD及其他安全公告平台和问题追踪系统(issue tracking system)中共收集的漏洞超过175,000个,涵盖200多种编程语言。
\item\textbf{Veracode\tocheck{链接},}该公司的漏洞数据库涵盖10多种编程语言相关的18,000多个漏洞,公开的漏洞信息包括:受漏洞影响的组件和版本范围、库修复说明、参考资料等。
\item\textbf{Snyk\footnote{https://snyk.io/vuln},}该公司声称:漏洞数据库\footnote{https://snyk.io/product/vulnerability-database/}是由经验丰富的安全研究团队持续维护,通过关注安全公告、Jira issue报告,Github commits等方式自动识别安全漏洞相关的报告。该公司的数据库涵盖超过10个编程语言生态系统,如:Maven、npm、Go、Composer等。该数据库提供漏洞的详细信息,包括:受漏洞影响的组件、版本范围、修复方法、参考链接等。
% \item \textbf{Gitlab Security} GitLab 咨询数据库是 Gitlab 依赖扫描器\footnote{https://docs.gitlab.com/ee/user/application\_security/dependency\_scanning/index.html} 的基础。目前涵盖了 6000 多个 CVE 条目和 8 个生态系统(即 Nuget、Conan、Maven 等)。提供了详细信息,例如描述、受影响的组件和版本、解决方案和参考。
\end{itemize}

进一步调研后发现,这四家公司中某些公司并未公开漏洞数据库,或是公开的漏洞信息中不包含用于修复漏洞的补丁信息,这将无法达成研究目标。最终,本文选定Veracode和Snyk的漏洞数据库作为研究对象,下文中简称为:$DB_A$和$DB_B$。


\subsection{广度数据集构建}
为了评估漏洞数据库中补丁的缺失程度以及不同数据库间补丁的不一致性(即:RQ1和RQ2),本文基于$DB_A$和$DB_B$构建了一个开源软件漏洞的广度数据集用以实验分析。截至2020年4月7日,分别从$DB_A$和$DB_B$中分别获取了\tocheck{8,630}和\tocheck{5,858}个CVE漏洞。

\subsection{深度数据集构建}
为了表征漏洞补丁的类型、映射关系以及尽可能准确地评估补丁信息的准确性(即:RQ3、RQ4和RQ5),本文还基于$DB_A$和$DB_B$的数据,构建了一个开源软件漏洞的深度数据集。该数据集的漏洞数量少于广度数据集,但每个漏洞都包含由人工确认的补丁信息。%为了确保漏洞补丁信息完整性和准确性,其中所有漏洞的补丁信息都是由人工识别。

在该深度数据集的构建过程中,为了确保数据集能够涵盖足够多的漏洞用以实验评估,但又不至于在人工识别补丁的阶段产生难以完成的工作量,本文仅将在$DB_A$和$DB_B$都含有补丁信息的漏洞列入该深度数据集,最终,该深度数据集共包含\tocheck{1,417}个CVE漏洞。

然后,对于该深度数据集中的每个CVE漏洞,首先分别由两位研究人员通过分析$DB_A$和$DB_B$数据库报告的补丁、查看NVD中的漏洞描述和参考链接信息以及搜索GitHub代码仓库的提交历史和其他网络资源等方式,独立得找到其补丁信息;之后,对比由两位研究人员独立查找得到得补丁信息,对于补丁结果不一致的漏洞,两位研究人员再一起分析讨论直到达成共识。这两位研究人员分别是本文作者和与本文作者同课题组的学生。由于公开的信息有限,\tocheck{1,417}个CVE漏洞中的\tocheck{122}个CVE漏洞无法找到补丁信息,比如:漏洞CVE-2016-3942在NVD中没有漏洞报告,但$DB_A$和$DB_B$将 jsrender@f984e1\cite{jsrender}标识为其补丁,两位研究人员无法确认该补丁信息的准确性。最终,该深度数据集共包含了\tocheck{1,295}个CVE漏洞。


\begin{figure*}[!t]
    \centering
    \begin{subfigure}[b]{0.45\textwidth}
    \centering
    \includegraphics[scale=0.46]{res/rq0-year.pdf}
    %\vspace{-5pt}
    \caption{漏洞年份分布统计}\label{fig:rq0-year}
    \end{subfigure}
    \begin{subfigure}[b]{0.45\textwidth}
    \centering
    \includegraphics[scale=0.46]{res/rq0-language.pdf}
    %\vspace{-5pt}
    \caption{程序语言分布统计}\label{fig:rq0-language}
    \end{subfigure}
    %\vspace{-20pt}
    \caption{数据集中漏洞年份及程序语言分布统计}\label{fig:dataset}
\end{figure*}


本文还进一步分析了该深度数据集中\tocheck{1,295}个CVE开源软件漏洞的年份和程序语言分布情况,以评估该数据集是否具有代表性。如图\ref{fig:rq0-year}所示,CVE的数量逐年增加,这与Snyk的报告\cite{Snyk-report}一致。此外,本文通过分析补丁中更改的源文件类型来确定CVE的编程语言。如图\ref{fig:rq0-language}所示,深度数据集中的CVE漏洞涵盖了七种较为常用的程序语言,具有较好的语言多样性。因此,可以认为该深度数据集对于开源软件漏洞数据库具有较好的代表性。


\section{RQ1:覆盖率分析}\label{sec:coverage}
\begin{figure}[h]
    \centering
    \begin{subfigure}[b]{0.45\textwidth}
    \centering
    \includegraphics[scale=0.98]{res/rq1-CVE-IDs-VS.pdf}
    %\vspace{-5pt}
    \caption{开源软件漏洞}\label{fig:rq1-cves}
    \end{subfigure}
    \begin{subfigure}[b]{0.45\textwidth}
    \centering
    \includegraphics[scale=0.98]{res/rq1-CVE-IDs-Patches-VS.pdf}
    %\vspace{-5pt}
    \caption{含补丁信息的开源软件漏洞}\label{fig:rq1-cves-with-patches}
    \end{subfigure}
    %\vspace{-10pt}
    \caption{$DB_A$与$DB_B$间数据交集}\label{fig:intersection}
\end{figure}


如图\ref{fig:rq1-cves}所示,$DB_A$和$DB_B$数据库中共有的CVE漏洞为\tocheck{4,418}个,同时$DB_A$和$DB_B$分别包含\tocheck{4,212}和\tocheck{1,440}个特有的CVE漏洞;如图\ref{fig:rq1-cves-with-patches}所示,$DB_A$中\tocheck{3,607(41.8\%)}的CVE漏洞含有补丁信息,$DB_B$中\tocheck{2,412(41.2\%)}的CVE漏洞含有补丁信息;$DB_A$和$DB_B$数据库共有\tocheck{10,070}个开源软件CVE漏洞,而其中仅有\tocheck{4,602(45.7\%)}的漏洞提供了补丁信息。

由此可见,数据库$DB_A$和$DB_B$中开源软件漏洞的补丁覆盖率都较低,分别为41.8\%和41.2\%,漏洞补丁缺失的情况较为普遍。%而且可以看出,不同的漏洞数据库对\congyingEdit{OSS}漏洞的覆盖范围不同。
同时,这也体现出自动化补丁查找方法的必要性,可用于填补数据库中缺失的补丁信息。


\section{RQ2:一致性分析}\label{sec:consistency}

\begin{table}[!t]
    \centering
    \footnotesize
    \caption{补丁一致性结果}\label{table:consistency}
    %\vspace{-10pt}
    \begin{tabular}{|*{1}{C{5.0em}}|*{1}{C{5.0em}}*{1}{C{5.5em}}*{1}{C{5.5em}}|*{1}{C{5.0em}}*{1}{C{5.0em}}*{1}{C{5.0em}}|}
    \noalign{\hrule height 1pt}
    \multirow{2}{*}{补丁一致} & \multicolumn{3}{c|}{存在性不一致} & \multicolumn{3}{c|}{内容不一致} \\\cline{2-4}\cline{5-7}
     & 总数 & 无漏洞信息 & 无补丁信息 & 总数 & 包含关系 & 非包含关系 \\\noalign{\hrule height 1pt}
    % \multirow{2}{*}{Cons.} & \multicolumn{3}{c|}{Existence Inconsistency} & \multicolumn{3}{c|}{Content Inconsistency} \\\cline{2-4}\cline{5-7}
    % & Total & No CVE & No Patch & Total & Inclusion & Difference \\\noalign{\hrule height 1pt}
    907 (19.7\%) & 3,185 (69.2\%) & 1,392 (30.2\%) & 1,793 (39.0\%) & 510 (11.1\%) & 176 (3.8\%) & 334 (7.3\%)\\
    % $DB_{A}$ vs. $DB_{C}$ & 3,659 & 73 (2.0\%) & 3,540 (96.7\%) & 2,523 (69.0\%) & 1,017 (27.8\%) & 46 (1.3\%) & 15 (0.4\%) & 31 (0.8\%) \\
    % $DB_{B}$ vs. $DB_{C}$ & 2,490 & 75 (3.0\%) & 2,397 (96.3\%) & 1,687 (67.8\%) & 710 (28.5\%) & 18 (0.7\%) & 7 (0.3\%) & 11 (0.4\%)\\
    \noalign{\hrule height 1pt}
    \end{tabular}
\end{table}

为了分析两个数据库之间的补丁信息一致性情况,本节主要关注带有补丁的CVE漏洞,即:图\ref{fig:rq1-cves-with-patches}中的CVE漏洞。考虑到漏洞补丁的个数可能不唯一(即:可能为一组补丁集),所以仅当两个数据库针对同一漏洞提供的补丁集完全相同时,才判定为补丁信息一致。本节将补丁信息不一致分为存在性不一致和内容不一致两种情况。前者是指某一个数据库为该CVE漏洞提供了补丁信息,而另一个数据库却不存在该CVE信息,或是存在该CVE却不存在相关补丁信息;后者是指两个数据库都存在该CVE的补丁信息,但它们的补丁集并不完全一致,分为包含关系或非包含关系的不一致。这两种情况分别反映了出数据库$DB_A$和$DB_B$中开源软件漏洞及其补丁信息的不完整性,以及漏洞补丁信息可能是不准确的


表\ref{table:consistency}中展示了补丁一致性分析的结果。其中,第一列为在$DB_A$和$DB_B$中具有一致补丁集的CVE数量(907,19.7\%),第二至四列为补丁存在性不一致的CVE数量(3,185,69.2\%),最后三列为都存在补丁信息但补丁集内容不一致的CVE数量(510,11.1\%)。可以发现:\tocheck{4,602}个CVE中,(1)只有\tocheck{907(19.7\%)}的漏洞在$DB_A$和$DB_B$中有一致的补丁信息;(2)超过三分之二(即:\tocheck{3,185(69.2\%)})的CVE漏洞在数据库$DB_A$和$DB_B$中存在补丁信息不一致的情况,其中\tocheck{1,392(30.2\%)}的CVE漏洞不在$DB_{A}$或$DB_{B}$中,\tocheck{1,793(39.0\%)}的CVE漏洞都存在于$DB_{A}$和$DB_{B}$中但在某一数据库中无补丁信息;(3)\tocheck{510(11.1\%)}的CVE漏洞补丁信息都存在于$DB_{A}$和$DB_{B}$中补丁集内容不一致,其中,\tocheck{176(3.8\%)}CVE的来自于某一个数据库的补丁集包含来自另一个数据库的补丁集,\tocheck{334(7.3\%)}CVE的来自$DB_{A}$和$DB_{B}$的补丁集即不同也不包含。

这些结果表明,$DB_A$和$DB_B$间存在较多的补丁信息不一致情况,进而表明数据库中补丁信息的准确性也需要进一步评估。


\section{RQ3:补丁类型分析}\label{sec:type}

通过人工手动查找,深度数据集中\tocheck{1,295}个CVE漏洞共收集到\tocheck{3,043}个补丁。具体来说,可能是因为GitHub在开源软件中被广泛使用,\tocheck{2,852(93.7\%)}的补丁都是为GitHub commit形式,\tocheck{136(4.5\%)}的补丁为SVN commit形式,且仅有\tocheck{55(1.8\%)}的补丁为来自其他Git平台的commit形式。

此外,从CVE的角度统计来看,\tocheck{1,295}个CVE中\tocheck{1,202(92.8\%)}的CVE有GitHub commit类型的补丁,\tocheck{4(0.3\%)}的CVE有SVN commit类型的补丁。由于由于很多项目是从SVN切换为Git管理,\tocheck{48(3.7\%)}的CVE即有GitHub commit又有SVN commit类型的补丁。只有\tocheck{30(2.3\%)}的CVE的补丁全为来自其他Git平台的commit形式。这些结果表明,开源软件漏洞的补丁类型主要为GitHub和SVN commit。

\section{RQ4:补丁映射分析}\label{sec:cardinality}
\begin{figure}[!t]
\centering
\includegraphics[scale=0.5]{res/rq4-cardinality.pdf}
\vspace{-10pt}
\caption{Mapping Cardinalities between CVEs and Patches}\label{fig:rq4-cardinality}
\end{figure}

基于深度数据集中\tocheck{1,295}个CVE漏洞及其补丁数据,本节将分析开源漏洞及其补丁间在数量上的映射关系。本文将CVE与其补丁之间的映射关系分为三种类型,即:一对一、\tocheck{一对一组}及一对多。

一对一是指一个补丁来修复某个漏洞,后文中简记为:\textit{SP}(Single Patch)。如图\ref{fig:rq4-cardinality}所示。\tocheck{567(43.8\%)}的CVE与其补丁具有一对一的映射关系(\textit{SP})

\tocheck{一对一组}是指CVE与其补丁在数量上非一对一关系,一个漏洞有多个commit信息,然而,这些commit都是等效的,任何一个补丁都足以修补漏洞,后文中简记为:\textit{MEP}(Multiple Equivalent Patch)。等效的补丁是指代码变更完全一样的两个commit,主要有两个原因:(1)通过GitHub中的Pull Request功能修补CVE,此时拉取请求提交(requested commit)和合并提交(merged commits)是该漏洞的等效补丁集。例如,python-jose@89b463\cite{python-jose-1}是拉取请求提交(requested commit),python-jose@73007d\cite{python-jose-2}是用于修复CVE-2016-7036的合并提交(merged commits),这两个commit是等效的。(2)一些开源软件的仓库是由SVN迁移到GitHub,因此,在同一软件的SVN和GitHub的代码仓库中分别有用于修补该CVE的commit,且这两处的commit中代码变更完全一样且完全等效的。例如,james-hupa代码仓库从SVN迁移到了GitHub,SVN commit james-hupa@1373762\cite{james-hupa-1}与GitHub commit james-hupa@aff28a\cite{james-hupa-2}是等效的。
如图\ref{fig:rq4-cardinality}所示,\tocheck{567(43.8\%)}的CVE与其补丁具有一对一的映射关系(\textit{SP}),\tocheck{195(15.1\%)}的CVE与其补丁为一对一组映射关系(\textit{MEP})

一对多是指CVE与其补丁在数量上为一对多的关系,即:多个非等效的补丁用来修复某个漏洞。如图\ref{fig:rq4-cardinality}所示,\tocheck{533(41.2\%)}的CVE与其补丁为一对多的映射关系,可以进一步分为三种类型: 

\begin{itemize}[leftmargin=*]
\item 一个CVE是通过一个分支中的多个独立commit来修复的。这是因为该CVE较难修复需多次提交,或是后期发现初始的补丁不足以修复漏洞便追加了补丁。后文中简记为:\textit{MP}(Multiple Patch),占比\tocheck{7.8\%(101)}。例如,CVE-2017-17837由三个commit deltaspike@4e2502\cite{deltaspike-1}、deltaspike@72e607\cite{deltaspike-2}和deltaspike@d95abe\cite{deltaspike-3}修复。
\item 一个CVE由多个分支中的多个补丁集修复。这是因为该漏洞影响开源软件的多个版本,每个版本又都在单独的分支上维护。后文中简记为:\textit{MB}(Multiple Branches),占比\tocheck{ 28.7\%(372)}。例如,CVE-2019-19118影响了django框架的2.1.x、2.2.x、3.0.x和3.2.x版本,commit django@103ebe\cite{django-1}、django@36f580\cite{django-2}、django@092cd6\cite{django-3}和django@11c5e0\cite{django-4}分别修复了受影响的四个版本分支中,其中,commit django@103ebe与其他commit的代码变更并不相同。
\item 一个CVE由多个存储库中的多个补丁集修复。这是因为CVE影响了多个开源软件或一个开源库的多个版本,而这些版本是分布在独立的代码仓库中维护。文中简记为:\textit{MR}(Multiple Repositories ),占比\tocheck{ 4.6\%(60)}。例如,CVE-2016-5104影响了libimobiledevice和libusbmuxd两个开源软件,commit libimobiledevice@df1f5c\cite{libimobiledevice}和libusbmuxd@4397b3\cite{libusbmuxd}分修复了受影响的两个开源库。

\end{itemize}

这些结果展示了CVE及其补丁之间映射关系的多样性。在后文设计自动化补丁查找方法时,应充分考虑该特征。

\section{RQ5:补丁准确性分析}\label{sec:accuracy}
\begin{table}[!t]
    \centering
    \footnotesize
    \caption{$DB_A$和$DB_B$补丁准确性评估结果}\label{table:accuracy}
    %\vspace{-10pt} 
    % \begin{tabular}{|*{1}{C{4.6em}}|*{1}{C{3.4em}}|*{3}{C{2.0em}}|*{3}{C{2.0em}}|}
    \begin{tabular}{|c|c|ccc|ccc|}
    \noalign{\hrule height 1pt}
    \multirow{2}{*}{映射类型} & \multirow{2}{*}{数量} &  \multicolumn{3}{c|}{$DB_A$} & \multicolumn{3}{c|}{$DB_B$} \\\cline{3-8}
    & & Pre. & Rec. & F1 & Pre. & Rec. & F1 \\
    \noalign{\hrule height 1pt}
    1:1 (SP) & 567       & 0.908 & 0.915 & 0.910   & 0.900 & 0.921 & 0.906   \\
    1:$i$ (MEP) & 195    & 0.935 & 0.898 & 0.902  & 0.924 & 0.909  & 0.906   \\
    1:$n$ (MP) & 101     & 0.923 & 0.483 & 0.616  & 0.911 & 0.520 & 0.638    \\
    1:$n$ (MB) & 372     & 0.941 & 0.510 & 0.620  & 0.932 & 0.436 & 0.555    \\
    1:$n$ (MR) & 60      & 0.913 & 0.610 & 0.695  & 0.964 & 0.526 & 0.636   \\\hline
    Total & 1,295       & 0.923 & 0.748 & 0.793  & 0.917 & 0.730 & 0.771     \\
    \noalign{\hrule height 1pt}
    \end{tabular}
\end{table}

本文使用精度(precision)、召回率(recall)和F1值(F1-score)作为评估补丁准确性的指标。对于具有两个等效补丁的CVE,报告两个等效补丁之一的数据库的精度和召回率都为1,而报告两个等效补丁之一和另一个不相关补丁的数据库的精度为1和召回率为0.5。

如表\ref{table:accuracy}所示,分解了两个数据库在映射基数方面的准确性结果。
第一列为CVE与补丁的映射类型,第二列为每种映射类型的CVE数量,最后六列分别为数据库$DB_A$和$DB_B$中CVE补丁的准确率、召回率和F1值。其中,$DB_A$和$DB_B$对于\textit{SP}和\textit{MEP}类型的CVE可实现约90\%的精度和召回率。同时,对于\textit{MP}、\textit{MB}和\textit{MR}类型的CVE,可达到90\%以上的高精度,但约50\%的召回率。例如,对于漏洞CVE-2017-17837,$DB_A$和$DB_B$仅报告三个补丁中的一个;对于漏洞CVE-2019-19118,$DB_A$报告四个补丁中的两个,而$DB_B$仅报告四个补丁中的一个。

这些结果表明,漏洞数据库$DB_A$和$DB_B$经常会遗漏一些漏洞的补丁信息,尤其是对于具有多个补丁的CVE漏洞。对于安全服务用户来说,这会给漏洞的及时检测和修复带来较大挑战,这也反映出自动化补丁查找方法的必要性。         % 
\chapter{\tool --补丁定位方法}

本章将详细阐述\tool --一种基于多源知识的开源软件漏洞的补丁识别方法的设计。

\section{方法概述}
\begin{figure*}[h]
    \centering
    \includegraphics[scale=0.40]{res/overview.pdf}
    %\vspace{-10pt}
    \caption{\tool 方法概览}\label{fig:overview}
\end{figure*}

基于上文经验研究的发现,本文提出了一种名为\tool 的自动化方法来查找来源软件漏洞的补丁(Github Commit形式)。\tool 的基本思想是:漏洞的补丁信息(即:commit)会在与该漏洞相关的各种来源的漏洞公告、分析报告、讨论和解决的过程中被频繁提及和引用。

图\ref{fig:overview}展示了\tool 的方法概览。\tool 以漏洞的CVE标识符作为输入,最终返回其补丁信息。具体分为三步:首先,\tool 从多个\tocheck{信息源}(即:NVD、Debian\cite{debian}、RedHat\cite{redhat}和GitHub)为输入的CVE构建一个相关引用链接的信息网络。该步骤的目的是将CVE在报告、讨论和解决阶段的引用链接信息进行建模。这里将NVD视为主信息来源,将Debian、RedHat和GitHub视为次级信息来源,该级信息源可以进一步扩展。其次,\tool 从构建的参考链接网络中,选择中具有高连通性和高置信度的补丁节点(即:commit)作为该CVE的补丁。最后,\tool 通过搜索同一存储库中其他分支上的相关提交来扩展补丁集。该步骤目的是在CVE及其补丁之间建立潜在的一对多映射关系。在本章的其他小节中,将详细阐述每个步骤。

% \begin{itemize}[leftmargin=*]
% \item 首先,\tool 从多个\tocheck{信息源}(即:NVD、Debian\cite{debian}、RedHat\cite{redhat}和GitHub)为输入的CVE构建一个相关参考链接的网络。该步骤的目的是将CVE在报告、讨论和解决阶段的参考链接信息进行建模。这里将NVD视为主信息来源,将Debian、RedHat和GitHub视为次级信息来源,该级信息源可以进一步扩展。
% \item 其次,\tool 从构建的参考链接网络中,选择中具有高连通性和高置信度的补丁节点(即:commit)作为该CVE的补丁。
% \item 最后,\tool 通过搜索同一存储库中其他分支上的相关提交来扩展补丁集。该步骤目的是在CVE及其补丁之间建立潜在的一对多映射关系。
% \end{itemize}

\section{步骤一:构建多源引用信息网络}
\tool 的步骤一共包括三个子步骤,前两个子步骤为漏洞公告分析和引用分析,通过分析来自NVD、Debian和Red Hat三个信息源的漏洞公告构建初始引用信息网络;第三个子步骤为信息增强,通过从GitHub搜索相关提交链接来扩充信息网络。

\subsection{公告分析}
首先,\tool 初始化信息网络,将输入的CVE-ID设置为根节点,然后再添加三个漏洞公告源节点(即:NVD、Debian和Red Hat)作为root的子节点。这些公告源节点用于追溯最终选定的补丁节点的来源。

\begin{figure}[h]
    \centering
    \includegraphics[scale=0.68]{res/network-example.pdf}
    %\vspace{-10pt}
    \caption{样例CVE-2017-11428的多源信息网络}\label{fig:example}
\end{figure}

\begin{exmp}
图\ref{fig:example}为样例CVE-2017-11428的完整的多源引用信息网络。 其中,顶层显示根节点,第二层显示公告来源节点。
\end{exmp}

然后,\tool 基于CVE-ID分别获取NVD、Debian和Red Hat的漏洞公告。其中,NVD平台以JSON形式按年份提供所有漏洞的结构化数据\cite{nvd-feed},\tool 通过下载并解析相应的JSON文件即可获得NVD中该CVE的信息。Debian平台的漏洞公告存储在仓库\cite{debian-repo}中,\tool 可直接从中解析得Debian提供的该CVE。Red Hat平台提供了 WebService API\cite{redhat-api}服务,\tool 可以直接使用该服务来检索Red Hat平台的漏洞公告。值得注意的是,分析发现:Debian会跟踪NVD上的所有CVE漏洞,而Red Hat仅会跟踪部分CVE漏洞。

\tool 从每个信息源的漏洞公告中提取引用信息(即:URL),并将它们添加为相应公告源节点的子节点。对于NVD公告,\tool 从“references”字段中提取出相关URL;类似地,对于Debian公告,\tool 会从“Notes”字段中提取出相关URL;对于Red Hat公告,\tool 使用正则表达式从评论区“comments”字段中提取出相关URL,这是因为开发人员会在评论区讨论和记录漏洞的解决过程,并可能列出对补丁信息。

\begin{exmp}
如图\ref{fig:example}中的第三层所示,对于CVE-2017-11428,NVD中包含了两个引用链接。链接一\tocheck{引用}是对描述此漏洞细节的博客的引用,链接一\tocheck{引用}是对该漏洞的第三方公告的引用。同时,这两个引用链接也包含在Debian公告中,不过该公告还包含对修复此漏洞的GitHub提交链接ruby-saml@048a54\cite{ruby-saml-1}的引用。此外,Red Hat平台并未收录该CVE。
\end{exmp}

\tool 将引用链接节点分为三种类型:补丁节点(Patch Node)、\tocheck{问题节点}(Issue Node)和\tocheck{混合节点}(Hybrid Node)。这里区分出补丁节点,是因为该方法的目标是为CVE找到补丁。区分出问题节点是因为开发人员常常会在问题追踪系统(Issue Tracker)中讨论该issue的解决方案并引用补丁链接信息;此外,问题追踪系统(Issue Tracker)中的报告会为CVE分配一个标识符(issue-id),开发人员常常会将issue-id写入补丁提交信息(即:commit message)中。未识别为补丁或问题节点的引用链接节点将被视为混合节点,它们多为博客、第三方漏洞公告等网页链接。

%受经验研究中补丁类型分析结果启发(Sec.\ref{sec:type}),
对于补丁节点的识别,如果URL链接中包含“git”字段且可通过正则表达式匹配到commit-id,则该链接为SVN commit形式的补丁节点;如果URL链接中包含“svn”字段且可通过正则表达式匹配到commit-id,则该链接为SVN commit形式的补丁节点。

对于问题节点的识别,如果URL链接中包含``/github.com/''和``/issues/'',则该链接为GitHub issue形式的问题节点;如果URL链接中包含``/github.com/''和``/pull/'',则该链接为GitHub pull request形式的问题节点;如果URL链接中包含``bugzilla"、``jira"、``issues"、``bugs"、``tickets"和``tracker"中的某一个字段且可通过正则表达式匹配到issue-id,\tocheck{则该链接为通常issue tracker形式的问题节点}。

\begin{exmp}
如图\ref{fig:example}中的第三层所示,NVD和Debian公告中包含的两个引用链接被标识为混合节点(即:图中的两个紫色节点),仅有一个在Debian公告中的引用链接被识别为补丁节点(即:图中的红色节点)。
\end{exmp}

\subsection{引用分析}

\subsection{信息增强}

\section{步骤二:精选补丁节点}

\section{步骤三:补丁扩增}
\chapter{实验验证及结果分析}

本章节实验评估。

\section{实验设计}

\section{RQ6:准确性验证}

\section{RQ7:削弱性分析}

\section{RQ8:敏感度分析}

\section{RQ9:通用性分析}

\section{RQ10: 实用性能分析}

\section{讨论}

% \include{src/c6_related.tex}
\chapter{总结}

% 本章节总结与展望。

\section{本文总结}
本文首先从补丁覆盖率、补丁一致性、补丁类型、漏洞与补丁之间的映射基数以及补丁准确性五个方面进行了一项经验研究,以了解两个工业漏洞数据库中开源软件漏洞补丁的质量和特征。

受经验研究结果的启发,本文提出一种名为\tool 的自动化补丁查找方法,以从多个知识源中查找开源软件漏洞的补丁。本文还设计了大量实验验证了\tool 的准确性、通用性和实用性。\tool 的源代码和所有实验数据已经在\url{https://patch-tracer.github.io}网站上发布。

% \section{未来展望}

% 未来展望未来展望未来展望未来展望未来展望未来展望未来展望未来展望

% 后置部分包含参考文献、声明页(自动生成)等
\backmatter

\printbibliography                    % 参考文献列表
\chapter{致谢}

三年学业已近尾声,值此毕业之际,向三年来给予我支持、鼓励及帮助的老师、同学、朋友及恋人表达真挚的感谢。

首先,我要感谢复旦大学计算机学院院长、软件工程实验室彭鑫教授。三年前,我还在扬州,因为推免事项,与彭老师一次次长时间电话交流的场景依然历历在目。最终,我很荣幸能得到彭老师的认可,进入复旦大学软件工程实验室进行研究生阶段的学习。在软工实验室学习的三年,收获颇多,是我人生中一段重要、难忘的时光。

我要感谢我的指导老师陈碧欢教授。在陈碧欢老师的指导下,我得到很多锻炼和学习的机会,参与上汽、吉利和华为等企业的合作项目,参与三篇CCF-A/B类论文工作;同时,我也以第一作者身份完成一篇FSE投稿论文。最要感谢陈老师对我个性的包容,我是一个不太老实、不会按部就班做事的学生,给陈老师带来不少额外的工作,十分感谢老师的包容与支持。我也要感谢实验室的伙伴们,黄凯锋、王颖、施博文、吴帅、谢隽丰、吴舒仪、潘灵号等同学在科研、学习以及生活上给我莫大帮助。

感谢我的家人,永远给予我最大的支持和肯定。因为有他们,我敢于在任何时候、任何事情上大胆尝试,不怕失败。

最后,感谢我自己,三年来,吃一堑,长一智。山外有山,人外有人,再接再厉。承担责任,享受人生,无问西东。  % 致谢
\chapter{初审意见修改说明}

衷心感谢老师们对本文耐心、严谨的评审,以及提出的宝贵修改意见。修改说明如下:


\begin{enumerate}[]
  \item \textbf{初审1}\\
  格式问题:
  \begin{itemize}
    \item \strong{扉页(提交学院和学校盲审版时去掉指导小组成员信息) ×(缺页);}\\
      已补充“指导小组成员”信息页,并在提交学院和学校盲审版时将其屏蔽。
    \item \strong{摘要  中英文摘要用罗马数字编页码,从I开始;}\\
      已重新编排页码,中文摘要页现从I开始。
    \item \strong{正文中各级标题 不同级标题之间不要直接相连,其间用一段总括的句子隔开;}\\
      文中,子标题“2.1 背景知识”、“2.2 相关工作”、“3.1 研究设计”之前及“第六章 总结与展望”之后,已添加一段总括的句子。
    \item \strong{独创性声明和论文使用授权声明(提交学院和学校盲审版时去掉这部分) 缺少该页}\\
      在提交学院和学校盲审版时,独创性声明和论文使用授权声明已屏蔽,未在论文中显示该页。盲审结束后,会补充该页。
  \end{itemize}

  正文内容: \\
  正文内容完整详细,逻辑通顺,但有一些小的细节问题。
  \begin{itemize}[]
      \item \strong{第九页 语句不通顺,‘出现了许多第三方平台致力于收集并维护所有漏洞公告信息’—>‘出现了许多致力于收集并维护所有漏洞公告信息的第三方平台’。}\\
        文中已更改为“出现了许多致力于收集并维护所有漏洞公告信息的第三方平台”。
      \item \strong{第十八页: 3.3 RQ1:补丁覆盖率分析各种比率写的比较混乱,例如对于包含10070个开源软件漏洞的广度,仅有1417个漏洞含有补丁(不是2190+1417+995?),补丁覆盖率为32\%。}\\
        此处为笔误,文中已更改为“对于包含 10,070 个开源软件漏洞的广度数据集,仅有4,602个漏洞含有补 丁,补丁覆盖率为 45.7\%。”。
      \item \strong{第十九页: 表3-1 无漏洞信息、无补丁(有歧义,应注释在某一数据库无漏洞信息、无补丁)。}\\
        表3-1中,已将“无漏洞信息“、”无补丁”分别更改为“某一数据库中无漏洞”、“某一数据库中无补丁”。
      \item \strong{第二十页: 语言不准确 RQ2:补丁一致性分析 1793(39\%)的漏洞都存在于$DB_A$和$DB_B$中但在某一数据库中无补丁--> 1793(39\%)的漏洞都存在于$DB_A$或$DB_B$中但在另一数据库中无补丁。}\\
        文中已更改为“1793(39\%)的漏洞都存在于$DB_A$或$DB_B$中但在另一数据库中无补丁”。
  \end{itemize}

  \item \textbf{初审2}\\
  文章针对当前漏洞研究中漏洞补丁管理中的不足,设计了相关统计实验进行研究总结,并根据经验研究结果针对已有漏洞补丁数据库的痛点,设计了 TRACER 工具来自动识别相关漏洞的漏洞补丁进行漏洞补丁数据库的补充,通过充足的实验数据证明了 TRACER 能有效提升漏洞补丁的覆盖率和准确率。在实用性、通用性等都表现了较高的实用价值。可应用于安全社区、工业和学术界。\\

  优点:文章研究内容具有一定的创新性、可用性,具有一定的学术价值。文章结构组织合理,逻辑清晰,行文流畅。实验充分。引用丰富。
  待改进地方: 
  \begin{itemize}[]
      \item \strong{本文缺少TRACER 测试时的硬件资源情况。例如运行的时间成本、物理环境或实现TRACER代码量等。}\\
        文中已添加章节“5.2 实验环境准备”,介绍了TRACER在测试时的硬件资源情况,包括运行的时间成本、物理环境及实现TRACER的代码量。具体内容为:“本文作者使用Python 3.6搭建实验平台、开发工具\tool ,\tool 中共包括6117行代码。实验评估中,\tool 运行在含有16核的Intel(R) Xeon(R) CPU @ 2.10GHz处理器、 64GB内存、Red Hat 4.8.5-39系统的服务器上。基于从广度数据集中随机抽样的200个漏洞,运行结果表明,\tool 在该实验环境下完成一次漏洞补丁定位的平均用时约为22分钟。”。
      \item \strong{部分标点格式不统一。例如: p53 中 6.1节 (3) 中冒号。}\\
        已调整p53中6.1节(3)中冒号的格式。
  \end{itemize}
  
\end{enumerate}




           % 评审意见修改

\end{document}
